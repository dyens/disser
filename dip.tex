\documentclass[a4paper,10pt]{article}
\usepackage[utf8x]{inputenc}
\usepackage[russian]{babel}
\usepackage{amssymb,amsfonts,amsmath,mathtext}
\usepackage{cite,enumerate,float,indentfirst}
\usepackage{graphicx}
\usepackage{graphics}
\usepackage{graphicx,amssymb}
%%%%%% DEFINITIONS %%%%%%%%%%%%%%%%%%%%%%%%%%%
\def\eps{\varepsilon}
\def\Dm{\widetilde{\cal D}_{\mu}}
\def\const{{\rm const\,}}
\def\D{{\cal D}}
\def\L{{\cal L}}
\def\A{{\cal A}}
\def\S{{\cal S}}
\def\E{{\cal E}}
\def\x{{\bf x}}
\def\p{{\bf p}}
\def\k{{\bf k}}
\def\q{{\bf q}}
\def\bfr{{\bf r}}
\def\h{{\bf h}}
\def\n{{\bf n}}
\def\bfx{{\bf x}}
\def\bfv{{\bf v}}
\def\bfu{{\bf u}}

\def\Dm{\widetilde{\cal D}_{\mu}}
\def\D{{\cal D}}
\def\L{{\cal L}}
\def\S{{\cal S}}
\def\bfr{{\bf r}}
\def\bfx{{\bf x}}
\def\bfv{{\bf v}}

%%%%%%%%%%%%%%%%%%%%%%
%opening
%opening
\title{Pots+Kraichnan}

\begin{document}

\maketitle


\section{Введение}\label{sec:Intro}

\section{Описание модели. Теоретико-полевая формулировка модели}
\label{sec:QFT}



\section{Канонические размерности, УФ-расходимости и перенормировка}
\label{sec:Reno}

Как известно, анализ ультрафиолетовых (УФ-) расходимостей
связан с анализом канонических размерностей.
Динамические модели типа (\ref{actionA}), (\ref{actionG}) и
(\ref{ActionA})--(\ref{ActionC}),
в отличии от статических, имеют два независимых масштаба: масштаб времени $T$ и масштаб длины $L$.
Таким образом каноническая размерность любой величины $ F $ (поле или параметр) характеризуется двумя числами, частотной размерностью $ d_{F}^{\omega}$
и импульсной $d_{F}^{k}$, которые определяются так, что $[F] \sim [T]^{-d_{F}^{\omega}} [L]^{-d_{F}^{k}}$ .
Эти размерности находятся из условий нормировки
\[ d_k^k=-d_{\bf x}^k=1,\ d_k^{\omega} =d_{\bf x}^{\omega }=0,\
d_{\omega }^k=d_t^k=0, \ d_{\omega }^{\omega }=-d_t^{\omega }=1 \]
и  требования, что каждый член функционала действия должен быть безразмерным (по отношению к импульсной и частотной размерности по отдельности).
Затем, основываясь на $d_{F}^{k}$ и $d_{F}^{\omega}$, можно ввести полную каноническую размерность
$d_{F}=d_{F}^{k}+2d_{F}^{\omega}$.

Канонические размерности для моделей (\ref{ActionA})--(\ref{ActionC})
приведены в таблице~\ref{table1}, в том числе и размерности для перенормированных параметров (без индекса ``0''), которые скоро будут введены.
Поля модели {\it A} отмечены индексом {\it A},
а поля  модели Грибова - {\it G}. Размерности параметров $\lambda_{0}$, $\tau_{0}$ и т.д. одинаковы в обеих моделях.
\begin{table}[H]
\caption{Канонические размерности в моделях (\protect\ref{ActionA})-(\protect\ref{ActionC}).}
\label{table1}
\begin{tabular}{|p{0.3cm}|c|c|p{0.5cm}|c|p{0.5cm}|p{0.5cm}|p{0.5cm}|p{0.7cm}|c|p{1cm}|}
\hline
$F$ & $\psi_{A}$ & $\psi_{A}^{\dag}$ & $\psi_{G}$, $\psi_{G}^{\dag}$ &
$ {\bf v} $ &   $\lambda_{0}$, $\lambda$ &
$\tau_{0}$, $\tau$ &  $m$, $\mu$,$\Lambda$ & $g_{0}^{2}$, $u_{0}$ &
$w_{0}$ & $u$, $w$, $\alpha$, $a_{0}$, $a$ \\
\hline
%\tableline
$d_{F}^{k}$ & $d/2-1$ & $d/2+1$ & $d/2$ & $-1$ & $-2$  & 2
& 1 &  $4-d$ & $\xi$ & 0 \\
\hline
$d_{F}^{\omega }$ & 0 & 0& 0 & 1 & 1 & 0 & 0 &  0 & 0 & 0 \\
\hline
$d_{F}$ & $d/2-1$ & $d/2+1$ &
$d/2$ & 1 & 0 & 2 & 1 &  $4-d$  & $\xi$ & 0 \\
\hline
\end{tabular}
\end{table}
Как уже говорилось в конце предыдущего раздела, обе полные модели логарифмичны (все константы связи одновременно безразмерные) при  $d=4$ и $\xi=0$.
Таким образом, УФ-расходимости в функциях Грина проявляются в виде полюсов по $\varepsilon = 4-d$, $\xi$  и их линейным комбинациям.
Полная каноническая размерность произвольной 1-неприводимой функции Грина $\Gamma = \langle\Phi \cdots \Phi \rangle _{\rm 1-ir}$
задается отношением  \cite{Book3}
\begin{equation}
d_{\Gamma }=d_{\Gamma }^k+2d_{\Gamma }^{\omega }= d+2-N_{\Phi }d_{\Phi},
\label{dGamma}
\end{equation}
 где
 $N_{\Phi}=\{N_{\psi},N_{\psi^{\dag}}, N_{v}\}$  число соответствующих полей, входящих в функцию  $\Gamma$,
при этом подразумевается суммирование по всем типам полей. Полная размерность $d_{\Gamma}$
в логарифмичной теории (такая, что $\varepsilon=\xi=0$) является формальным индексом УФ расходимости
 $\delta_{\Gamma}=d_{\Gamma}|_{\varepsilon=\xi=0}$. Поверхностные УФ расходимости, устранение которых требует контрчленов, могут присутствовать
 только в тех функций  $\Gamma$, для которых $\delta_{\Gamma}$ не отрицательно.

Из таблицы ~\ref{table1} и (\ref{dGamma}) находим
\begin{equation}
\delta_{\Gamma}= 6 - N_{\psi} - 3N_{\psi^{\dag}} - N_{v}
\label{IndeA}
\end{equation}
для модели {\it A} и
\begin{equation}
\delta_{\Gamma}= 6 - 2N_{\psi} - 2N_{\psi^{\dag}} - N_{v}
\label{IndeG}
\end{equation}
для модели Грибова.

В динамических моделях 1-неприводимые функции Грина, в которых нет полей
$\psi^{\dag}$, равны $0$. Поэтому необходимо рассматривать только те функции, в которых  $N_{\psi^{\dag}} \ge 1$.
Для модели Грибова, функции Грина, построенные только из полей $\psi^{\dag}$ так же обнуляются из-за симметрий
(\ref{symm}), (\ref{symmG}), поэтому будем рассматривать только функции, для которых еще и $N_{\psi}\ge1$
в (\ref{IndeG}). Для модели {\it A}, из-за симметрии отражения,
достаточно рассматривать в  (\ref{IndeA}) только функции с $N_{\psi^{\dag}}+N_{\psi}$ кратным $2$ . С помощью этих ограничений, анализ выражении  (\ref{IndeA}), (\ref{IndeG})
показывает, что в обеих моделях поверхностные УФ расходимости могут быть лишь в следующих 1-неприводимых функциях Грина (приведены сами функции,
индексы расходимости и возможные контрчлены):
\[ \langle \psi^{\dag} \psi \rangle \quad (\delta=2) \quad
 \quad \quad \quad \psi^{\dag}\partial_{t}\psi, \
\psi^{\dag}\partial^{2}\psi, \ \psi^{\dag}\psi, \]
\[ \langle \psi^{\dag} \psi v \rangle \quad (\delta=1) \quad
  \quad \quad  \quad \psi^{\dag} (v\partial) \psi, \
\psi^{\dag} (\partial v) \psi.  \]
Для модели {\it A}, поверхностные расходимости могут быть еще и в таких функциях:
\[ \langle \psi^{\dag} \psi\psi\psi \rangle \quad (\delta=0) \quad
  \quad  \quad \quad \psi^{\dag} \psi^{3}, \]
для модели Грибова:
\[ \langle \psi^{\dag} \psi\psi \rangle \quad (\delta=0) \quad
  \quad \quad  \quad \psi^{\dag} \psi^{2}, \]
\[ \langle \psi^{\dag} \psi^{\dag}\psi \rangle \quad (\delta=0) \quad
  \quad \quad \quad (\psi^{\dag})^{2} \psi. \]
Поверхностные расходимости в функциях
$\langle \psi^{\dag} \psi vv \rangle$ с $\delta=0$ и котрчленом $\psi^{\dag} \psi v^{2}$, появляющиеся по размерности в обеих моделях, исчезают из-за присутствия Галилеевой симметрии.
\footnote{Использование Галилеевой симметрии применимо для поля скорости, которое описывается уравнением Навье-Стокса,
и вообщем это не так для специальных моделей, описывающих поле скорости с Гауссовым распределением.
Однако, для нашей используемой модели это справедливо. Детальное обсуждение проблемы было проведено в работе \cite{Alexa}. И доказательство справедливости применимо и в наших случаях, как
для модели (\ref{ActionA}), так и для  (\ref{ActionG}).}

Все такие члены представлены в действии (\ref{ActionA}) и (\ref{ActionG}), так что наши модели мультипликативно перенормируемы.
Галилеева симметрия требует, чтобы контрчлены
$\psi^{\dag}\partial_{t}\psi$ и $\psi^{\dag} (v\partial) \psi $ входили в перенормированное действие только в форме Лагранжевой производной
 $\psi^{\dag}\nabla_{t}\psi$.
Таким образом, перенормированные действия могут быть записаны в виде
\begin{eqnarray}
\S_{A}^{R}(\Phi) = \S_{C}^{R}(\Phi) +
\lambda Z_{4}  (\psi^{\dag})^{2} -
u\mu^{\varepsilon} \lambda Z_{5}  \psi^{\dag} \psi^{3} /3!
\label{ActionAR}
\end{eqnarray}
для модели {\it A} и
\begin{eqnarray}
\S_{G}^{R}(\Phi) = \S_{C}^{R}(\Phi)
+ \frac{g\mu^{\varepsilon/2}\lambda}{2} \left\{ Z_{4}(\psi^{\dagger})^2\psi
- Z_{5} \psi^{\dagger}\psi^2  \right\}
\label{ActionGR}
\end{eqnarray}
для модели Грибова, где
\begin{eqnarray}
\S_{C}^{R}(\Phi) =  \psi^{\dag} \left\{
- Z_{1} \nabla_{t} + \lambda\left( Z_{2} \partial^{2}- Z_{3}\tau\right)
- a Z_{6} (\partial_{i}v_{i}) \right\} \psi +  \S(\bfv),
\label{ActionCR}
\end{eqnarray}
а $\S(\bfv)$ из (\ref{Sv}).

Здесь $\lambda$, $\tau$, $g$, $u$ и $a$ -- ренормированные аналоги
первоначальных затравочных констант  (которые теперь будут помечаться
индексом ``0''), и нормировочная масса $\mu $ -- добавочный произвольный
параметр ренормированной теории. Выражение (\ref{ActionAR})--(\ref{ActionCR})
эквивалентно мультипликативной ренормировке полей $\psi \to \psi Z_{\psi}$,
$\psi^{\dag} \to \psi^{\dag} Z_{\psi^{\dag}}$
и параметров:
\begin{eqnarray}
g_{0} = g \mu^{\varepsilon/2} Z_{g}, \quad u_{0} = g \mu^{\varepsilon} Z_{u}, \quad
w_{0} = w \mu^{\xi} Z_{w}, \nonumber \\
\lambda_{0} = \lambda Z_{\lambda}, \quad
\tau_{0} = \tau Z_{\tau},  \quad  a_{0} = a Z_{a}.
\label{Multy}
\end{eqnarray}
 Константы ренормировки $Z_{i}$ имеют полюса по  $\varepsilon$ и $\xi$, и зависят только от безразмерных параметров
 $u$, $w$, $\alpha$ и $a$.

Вследствие того, что $\S(\bfv)$ заданное (\ref{Sv}) не ренормируется, $D_{0}$ из (\ref{Kraich}) выражается через ренормированные параметры: $D_{0} = w_{0} \lambda_{0}  = w\lambda \mu^{\xi}$.
Параметры $m$ и $\alpha$ не ренормируются: $m_{0} = m$,
$\alpha_{0} = \alpha$.
Благодаря Галилеевой симметрии, оба слагаемых в ковариантной производной $\nabla_{t}$  перенормируются одной константой $Z_{1}$.
Вследствие чего, имеем соотношения:
\begin{eqnarray}
Z_{w}Z_{\lambda} =1, \quad Z_{m}= Z_{\alpha} = Z_{v} =1.
\label{RenD}
\end{eqnarray}
Сравнивая выражения (\ref{ActionA})--(\ref{ActionC}) и
(\ref{ActionAR})--(\ref{ActionCR}) получаем соотношения между константами ренормировки $Z_{1}$--$Z_{6}$ и $Z$ из (\ref{Multy}):
\begin{eqnarray}
Z_{1} = Z_{\psi} Z_{\psi^{\dagger}}, \quad Z_{2} = Z_{1}Z_{\lambda}, \quad
Z_{3} = Z_{2} Z_{\tau}, \quad Z_{6} = Z_{1} Z_{a}
\label{ZZ}
\end{eqnarray}
для обеих моделей,
\begin{equation}
Z_{4} = Z_{\lambda} Z_{\psi^{\dagger}}^{2}, \quad
Z_{5} = Z_{\lambda}Z_{u} Z_{\psi}^{3} Z_{\psi^{\dagger}}
\label{ZA}
\end{equation}
для модели {\it A} и
\begin{equation}
Z_{4} = Z_{g} Z_{\lambda} Z_{\psi^{\dagger}}^{2} Z_{\psi},
\quad Z_{5} = Z_{g} Z_{\lambda} Z_{\psi^{\dagger}} Z_{\psi}^{2}
\label{ZG}
\end{equation}
для модели Грибова.
Решая эти соотношения относительно констант перенормировки поля и параметров, имеем
\begin{equation}
Z_{\lambda} = Z_{1}^{-1} Z_{2}, \quad Z_{\tau} = Z_{2}^{-1} Z_{3}, \quad
Z_{a} = Z_{1}^{-1} Z_{6}, \quad
Z_{u} = Z_{1}^{-1} Z_{2}^{-2} Z_{4} Z_{5}
\label{ResoC}
\end{equation}
для обеих моделей (в моделе Грибова мы произвели замену $u=g^{2}$, $Z_{u}=Z_{g}^{2}$),
\begin{eqnarray}
Z_{\psi}= Z_{1}^{1/2}Z_{2}^{1/2}Z_{4}^{-1/2}, \quad
Z_{\psi}^{\dag}= Z_{1}^{1/2}Z_{2}^{-1/2}Z_{4}^{1/2}
\label{ResoA}
\end{eqnarray}
для модели {\it A} и
\begin{eqnarray}
Z_{\psi}= Z_{1}^{1/2}Z_{4}^{-1/2}Z_{5}^{1/2}, \quad
Z_{\psi}^{\dag}= Z_{1}^{1/2}Z_{4}^{1/2}Z_{5}^{-1/2},
\label{ResoG}
\end{eqnarray}
для модели Грибова.\\
Константы перенормировки находятся из требования, что функции Грина
перенормированной моделеи (\ref{ActionAR})--(\ref{ActionCR}), когда они пишутся в теминах перенормированных
переменных, должны быть УФ конечными (в нашем случае, конечные при $\varepsilon\to0$,
$\xi\to0$).
Константы  $Z_{1}$--$Z_{6}$ вычисляются прямо из диаграмм, а затем константы из (\ref{Multy}) находятся из соотношений
(\ref{ResoC})--(\ref{ResoG}). Для того, чтобы найти полный набор констант, необходимо рассматривать 1-неприводиме функции
 Грина, имеющие поверхностные расходимости. Все эти функции были перечисленные выше.
В однопетлевом приближении, они приведены на рисунке ~\ref{fig:DA} для модели {\it A} и на рисунке  ~\ref{fig:DG} для модели Грибова.
Сплошные линии со стрелками обозначают пропагатор (\ref{lines}), стрелка указывает на поле $\psi^{\dag}$.
Сплошные линии без стрелок соответствует пропагатору (\ref{lines2}), а волнистые линии обозначают пропагатор скорости $\langle vv \rangle_{0}$ , указанный в  (\ref{white}).
Внешние концы с входящей стрелки соответствуют полю $\psi^{\dag}$, а концы без стрелки соответствуют $\psi$.
Тройная вершина с одной волнистой линией соответствует множителю (\ref{VertexF}).
\begin{figure}[H]
\begin{center}
 \includegraphics [width=13cm]{./figg1.jpg}
 % fig1.jpg: 522x247 pixel, 100dpi, 13.26x6.27 cm, bb=0 0 376 178
\end{center}
\caption{\label{fig:DA}
Одно-петлевое приближение для 1-неприводимых функций Грина в  моделе (\protect\ref{ActionA}).}
\end{figure}

\begin{figure}[H]
\begin{center}
%\includegraphics[width=11cm]{PATT.pdf}
\includegraphics[width=13cm]{./figg2.jpg}
\caption{\label{fig:DG}
 Одно-петлевое приближение для 1-неприводимых функций Грина в  моделе  (\protect\ref{ActionG}).}
\end{center}
\end{figure}

Минус перед первым слогаемым для функции $\langle \psi^{\dag} \psi \rangle_{1-ir}$ появляется в следствии уравнения Дайсона.
Первые и последние диаграммы в функциях
$\langle \psi^{\dag} \psi \psi  \rangle_{1-ir}$ и
$\langle \psi^{\dag} \psi^{\dag} \psi \rangle_{1-ir}$ для модели Грибова
имеют раличный вид (зеркально отражены), но их вклад в константы ренормировки одинаков, и поэтому мы ставим перед ними множитель 2.
Более того, последние диаграммы для этих функций имеют замкнутый цикл пропагаторов $\langle \psi \psi^{\dag} \rangle_{0}$ и поэтому равны $0$.
По этой же причине, последние диаграммы в выражении $\langle \psi^{\dag} \psi \psi \psi  \rangle_{1-ir}$ для модели {\it A}
и $\langle \psi^{\dag}\psi {\bf v} \rangle_{1-ir}$ для обеих моделей обнуляются.
Все элементы диаграмм должны быть выражены в перенормированных переменных с помощью соотношений  (\ref{Multy})--(\ref{ZZ}).
В однопетлевом приближении,  $Z$  в первых слогаемых рисунков  \ref{fig:DA} и  \ref{fig:DG}  должны быть записанны   до первого порядка малости по $u= g^{2}$
и $w$, в то время как в диаграммах они должны быть просто заменены единицами $Z_{i} \to 1$.
Таким образом, переход к перенормированным переменным в диаграммах достигается за счет простой замены $\lambda_{0} \to \lambda$, $\tau_{0} \to \tau$,
$g_{0} \to g\mu^{\varepsilon/2}$ and $w_{0} \to w\mu^{\xi}$.
В расчетах мы использовали схему минимальных вычитаний (MS), в которой константы перенормировки выглядят как  $Z_{i}=1+...$  и имеют  особенностей  по $\varepsilon$ и $\xi$,
с коэффициентами зависящими только от безразмерных  перенормированных параметров $u$, $w$, $a$ и $\alpha$.
Однопетлевой расчет довольно схож с расчетом в несжимаемом случае ($\alpha=0$), который подробно обсуждался в  \cite{AIK} (см. также  \cite{AHH,Alexa}).
 И здесь мы приведем лишь результаты:
\begin{eqnarray}
Z_{1} = 1, \quad  Z_{2} = 1 - \frac{w}{4\xi}(3+\alpha), \quad
Z_{3} = 1 + \frac{u}{\varepsilon}, \nonumber \\
Z_{4} = 1 - \frac{w}{\xi} \alpha (a-1)^{2} , \quad
Z_{5} = 1 + \frac{3u}{\varepsilon} - \frac{3w}{\xi} \alpha a^{2},
\nonumber \\
Z_{6} = 1 + \frac{u}{\varepsilon}\, \frac{(4a-1)}{4a}
\label{ZoA}
\end{eqnarray}
для модели {\it A} и
\begin{eqnarray}
Z_{1} = 1 + \frac{u}{4\varepsilon}, \quad
Z_{2} = 1 + \frac{u}{8\varepsilon} - \frac{w}{4\xi}(3+\alpha), \quad
Z_{3} = 1 + \frac{u}{\varepsilon}, \nonumber \\
Z_{4} = 1 + \frac{u}{\varepsilon} - \frac{w}{\xi} \alpha (a-1)^{2} , \quad
Z_{5} = 1 + \frac{u}{\varepsilon} - \frac{w}{\xi} \alpha a^{2}, \nonumber \\
Z_{6} = 1 + \frac{u}{\varepsilon}\, \frac{(4a-1)}{8a}
\label{ZoG}
\end{eqnarray}
для модели Грибова.\\
Для упрощения полученных выражений, мы зделали в (\ref{ZoA})
и  (\ref{ZoG}) замену переменных  \[ u \to u/16\pi^2, \quad w \to w/16\pi^2. \] Здесь и далее
они будут  обозначатся теми же символами  $u$ и $w$.
Для $w=0$, были уже хорошо известны  одно-петлевые результаты в моделях (\ref{actionA}),
(\ref{actionG}). С точностью до обозначений они получились и в наших расчетах. Для несжимаемой жидкости (случай $\alpha=0$) выражения (\ref{ZoG}) так же находятся в соответствии
с полученными ранее результатами в работе \cite{AIK}. Более того, в первом порядке выражения (\ref{ZoG}) удовлетворяют точным соотношениям
\begin{eqnarray}
Z_{i} (a) = Z_{i} (1-a) \quad {\rm for} \quad i=1,2,3; \nonumber \\
Z_{4} (a) = Z_{5} (1-a), \quad
Z_{1} (a) -a Z_{6}(a) = (1-a) Z_{6}(1-a),
\label{GSymm}
\end{eqnarray}
которые являются следствием симметрий (\ref{symm}), (\ref{symmG}) для модели Грибова. Эти соотношения играют важную роль в анализе ИК-притягивающих точек модели
(\ref{ActionG}).

\section{Уравнение ренормгруппы} \label{sec:RGE}

Напомним вывод уравнений РГ; подробное изложение можно найти в \cite{Zinn,Book3}.

Уравнение ренормгруппы записывается для ренормированной
корреляционной функции $G_{R} =\langle \Phi\cdots\Phi\rangle_{R}$,
которая отличается от первоначальной (неренормированной) $G
=\langle \Phi\cdots\Phi\rangle$ только нормировкой и выбором
параметров и, следовательно, с равным правом может быть
использована для анализа критического поведения.
Соотношение $\S_{R} (\Phi,e,\mu) =
\S(\Phi,e_{0})$ между функционалами действия (\ref{ActionA})--(\ref{ActionC})  и
(\ref{ActionAR})--(\ref{ActionCR}) приводит к соотношению между функциями Грина:
\begin{equation}
G(e_{0},\dots) = Z_{\psi}^{N_{\psi}}
Z_{\psi^{\dagger}}^{N_{\psi^{\dagger}}} G_{R}(e,\mu,\dots).
\label{multi}
\end{equation}
Здесь, как обычно, $N_{\psi}$ и $N_{\psi^{\dagger}}$ -- числа
входящих в $G$ полей (напомним, что в нашей модели
$Z_{v}=1$);  $e_{0}=\{\nu_{0}, \tau_{0}, w_{0}, g_{0} \}$ -- набор
первоначальных параметров, и $e=\{ \nu, \tau, w, g \}$ -- их
ренормированные аналоги; точки символизируют остальные переменные
(время, кордината, импульс и т.д.)

Применим к обеим частям равенства (\ref{multi}) операцию
$\widetilde{\cal D}_{\mu}$ -- производную вида $\mu\partial_{\mu}$
при фиксированном наборе затравочных параметров $e_{0}$.
Это даст нам основное
дифференциальное уравнение ренормгруппы на функцию
$G^{R}(e,\mu,\dots)$
\begin{equation}
\left\{ {\cal D}_{RG} + N_{\psi}\gamma_{\psi} +
N_{\psi^{\dagger}}\gamma_{\psi^{\dagger}} \right\}
\,G^{R}(e,\mu,\dots) = 0, \label{RG1}
\end{equation}
где ${\cal D}_{RG}$ -- оператор  $\widetilde{\cal D}_{\mu}$,
выраженный в ренормированных переменных:
\begin{equation}
{\cal D}_{RG}\equiv {\cal D}_{\mu} + \beta_{u}\partial_{u} +
\beta_{w}\partial_{w} - \gamma_{\tau}{\cal D}_{\tau} -
\gamma_{\nu}{\cal D}_{\nu}.
\label{RG2}
\end{equation}
Здесь и далее мы пользуемся обозначением ${\cal D}_{x}\equiv
x\partial_{x}$ для любой переменной $x$, и
\begin{equation}
\gamma_{F}\equiv \widetilde{\cal D}_{\mu }\ln Z_{F}
\label{RGF1}
\end{equation}
-- аномальная размерность для любой величины (поля или параметра) $F$.
 В свою очередь, вводимые для безразмерных констант
$u$, $w$ и $a$ (зарядов), $\beta$-функции имеют вид:
\begin{eqnarray}
\beta_{u} \equiv \widetilde {\cal D}_{\mu} u = u\, (-\varepsilon-\gamma_{u}),
\nonumber \\
\beta_{w} \equiv \widetilde {\cal D}_{\mu} w = w\,(-\xi-\gamma_{w}),
\nonumber \\
\beta_{a} \equiv \widetilde {\cal D}_{\mu} a = -a\gamma_{a},
\label{betagw}
\end{eqnarray}
где вторые равенства следуют из определений и соотношений (\ref{Multy}).
Четвертая $\beta$ функция
\begin{eqnarray}
\beta_{\alpha}=\widetilde{\cal D}_{\mu}\alpha=-\alpha\gamma_{\alpha}
\label{Bal}
\end{eqnarray}
тождественно равна $0$, вследствие (\ref{exi}).
Аномальные размерности для данной константы перенормировки $Z_{F}$  легко получить из соотношений
 \begin{equation}
\gamma_{F} = \left( \beta_{u}\partial_{u}+\beta_{w}\partial_{w}
+\beta_{a}\partial_{a} \right)
\ln Z_{F} \simeq  - \left(\varepsilon\D_{u}+\xi\D_{w}\right) \ln Z_{F}.
\label{GfZ}
\end{equation}
В первом равенстве, мы использовали определение  (\ref{RGF1}), выражение (\ref{RG2})
для операции $\Dm$  в ренормированной переменных и тот факт, что $Z$ зависят
только от полностью безразмерных констант связи $u$, $w$ и $a$.
Во втором  соотношении, мы сохранили лишь главные члены в  $\beta$ функциях (\ref{betagw}). Этого достаточно в нашем приближении.
В первом порядке выражения (\ref{ZoA}), (\ref{ZoG}) для констант ренормировки имеют вид

\begin{equation}
Z_{F} = 1 + \frac{u}{\varepsilon} A_{F}(a,\alpha) + \frac{w}{\xi} B_{F}(a,\alpha).
\label{Zf}
\end{equation}
Подставляя (\ref{Zf}) в (\ref{GfZ}), получаем итоговое, УФ конечное выражение для аномальных размерностей:
\begin{equation}
\gamma_{F} = - u A_{F}(a,\alpha) - w B_{F}(a,\alpha)
\label{gift}
\end{equation}
 И в наших случаях имеем:
\begin{eqnarray}
\gamma_{1} = 0, \quad \gamma_{2} = w (3+\alpha)/4, \quad
\gamma_{3} = -u , \quad \gamma_{4} =w \alpha (a-1)^{2} , \nonumber \\
\gamma_{5} =  -3u +3 w \alpha a^{2}, \quad
\gamma_{6} =u(1-4a)/4a
\label{anomA}
\end{eqnarray}
для модели {\it A} и
\begin{eqnarray}
\gamma_{1} = -u/4 , \quad \gamma_{2} = -u/8 + w (3+\alpha)/4, \quad
\gamma_{3} = -u/2 , \nonumber \\
\gamma_{4} = -u + w \alpha (a-1)^{2}, \quad
\gamma_{5} = -u + w \alpha a^{2}, \quad
\gamma_{6} =u(1-4a)/8a
\label{anomG}
\end{eqnarray}
для модели Грибова.

Соотношения (\ref{ResoC})--(\ref{ResoG})
между константами ренормировки приводят к линейным соотношениям между соответствующими аномальными размернастями:
\begin{eqnarray}
\gamma_{\lambda} =  \gamma_{2} -\gamma_{1}, \quad
\gamma_{\tau} =  \gamma_{3} -\gamma_{2} , \nonumber \\
\gamma_{a} =  \gamma_{6} -\gamma_{1}, \quad
\gamma_{u} = -\gamma_{1}- 2\gamma_{2} +\gamma_{4} + \gamma_{5}
\label{aesoC}
\end{eqnarray}
для обеих моделей,
\begin{eqnarray}
2\gamma_{\psi}= \gamma_{1} +\gamma_{2} -\gamma_{4}, \quad
2\gamma_{\psi}^{\dag}= \gamma_{1} - \gamma_{2} + \gamma_{4}
\label{aesoA}
\end{eqnarray}
для модели {\it A} и
\begin{eqnarray}
2\gamma_{\psi}= \gamma_{1}- \gamma_{4} + \gamma_{5}, \quad
2\gamma_{\psi}^{\dag}= \gamma_{1}+ \gamma_{4} -  \gamma_{5}
\label{aesoG}
\end{eqnarray}
для модели Грибова.

Вместе с (\ref{anomA}), (\ref{anomG}), эти выражения дают конечные ответы в первом порядке для аномальных размерностей  полей и параметров.
Точные соотношения в  (\ref{RenD}) приводят к
\begin{eqnarray}
\gamma_{w} =-\gamma_{\lambda}, \quad
\gamma_{m} =\gamma_{\alpha} =\gamma_{v} = 0,
\label{exi}
\end{eqnarray}
А из (\ref{GSymm}) мы имеем
\begin{eqnarray}
\gamma_{i} (a) = \gamma_{i} (1-a) \quad {\rm for} \quad i=1,2,3;
\nonumber \\
\gamma_{4} (a) = \gamma_{5} (1-a), \quad
\gamma_{1} (a) -a \gamma_{6}(a) = (1-a) \gamma_{6}(1-a)
\label{SymmG}
\end{eqnarray}
для модели Грибова.

\section{Неподвижные точки} \label{sec:FPS}

Хорошо известно, что возможные скейлинговые  режимы перенормируемой теоретико-полевой
модели определяются асимптотикой системы обыкновенных дифференциальных уравнений для так называемой
инвариантной (бегущей) константы
\begin{eqnarray}
\D_s \bar g_{i}(s,g) = \beta_{i} (\bar g), \quad \bar g_{i}(1,g) = g_{i},
\label{Odri}
\end{eqnarray}
где  $s=k/\mu$, $k$ -- импульс, $g= \{g_{i}\}$ -- полный набор констант связи
и  $\bar g_{i}(s,g)$ -- соответствующие им инвариантные переменные.

Как правило, ИК ($s\to0$) и УФ ($s\to\infty$) поведение таких систем определяются неподвижными точками $g_{i*}$.
Координаты возможных неподвижных точкек находятся из требования, чтобы все   $\beta$ функции обращались в нуль:
\begin{eqnarray}
\beta_{i} (g_{*}) =0,
\label{fp}
\end{eqnarray}

 А тип данной неподвижной точки определяется матрицей
\begin{equation}
\Omega_{ij} = \partial\beta_{i}/\partial g_{j} |_{g=g^*}:
\label{OmegaDef}
\end{equation}
Лля ИК притягивающей неподвижной точки (которые нас и  интересуют)
матрицы $\Omega$ положительно определена, то есть вещественные части всех ее собственных значений положительны.
В нашех моделях, неподвижные точки для полного набора констант $u$, $w$, $a$, $\alpha$ должны определяться уравнениями
\begin{equation}
\beta_{u,w,a,\alpha} (u_{*},w_{*},a_{*},\alpha_{*}) = 0,
\label{points}
\end{equation}
гле $\beta$- функций определенные нами ранее.

Однако, в наших моделях аттракторы системы (\ref{Odri}) лежат на двумерных поверхностях в полном четырехмерном пространстве.
Во-первых, функция (\ref{Bal})  тождественно равна нулю, так что уравнение $\beta_{\alpha}=0$ не дает никаких ограничений на параметр  $\alpha$.
Поэтому удобно рассматривать аттракты систем (\ref{Odri}) в трехмерном пространстве $u$, $w$, $a$; их координаты, матрица (\ref{OmegaDef})
и критические индексы будут, вообще говоря, зависеть от свободного параметра $\alpha$.
Хотя общая картина аттракторов выглядит довольно одинакова для обеих моделей, мы будем рассматривать их отдельно.


\subsection{Скейлинговые режимы в моделе Грибова} \label{sec:GPS}
Однопетлевом выражения для $\beta$-функций в модели (\ref{ActionGR}), (\ref{ActionCR}) легко выводятся из определений (\ref{betagw}), отношений (\ref{aesoC}), (\ref{aesoG}) и (\ref{exi}) и
выражений (\ref{anomG}):
\begin{eqnarray}
\beta_{u} = u \left[ -\varepsilon+ 3u/2 + w(3+\alpha)/2 -w\alpha f(a) \right],
\nonumber \\
\beta_{w} = w \left[ -\xi +u/8 + w(3+\alpha)/4 \right],
\nonumber \\
\beta_{a} = u(2a-1)/8,
\label{betaG}
\end{eqnarray}
где, $f(a)=a^{2}+(a-1)^{2}$  имеет минимальное значение $f(1/2) =1/2$ при $a=1/2$. Для наших (\ref{betaG}) функций в моделе Грибова уравнения
(\ref{points}) имеют следующие решения:\\

(1)  Гауссова  (свободная) неподвижная точка: $u_{*}=w_{*}=0$, $a_{*}-\forall$.\\

(2) Точка  $w_{*}=0$, $u_{*}=2\varepsilon/3$, $a_{*}=1/2$, что соответствует читстой модели Грибова
(турбулентный перенос не имеет значения).\\

(3) Линия неподвижных точек
\begin{equation}
u_{*}=0, \quad w_{*}=4\xi/(3+\alpha), \quad  a_{*} - \forall,
\label{line3}
\end{equation}
соответствует  скалярному полю без самовзаимодействия.\\


(4) Самая интересная, не тривиальная точка, соответсвует новому режиму (классу универсальности), где важно как само Грибовское взаимодействие так и перемешивание :
\begin{equation}
u_{*} = \frac{4\,[\varepsilon(3+\alpha)-6\xi]}{3(5+2\alpha)}, \quad
w_{*} = \frac{8\,[-\varepsilon/4+3\xi]}{3(5+2\alpha)}, \quad a_{*}=1/2.
\label{wu4}
\end{equation}
\\
Напомним, что  $\alpha$ рассматривается здесь как свободный параметр, от которого зависят координаты неподвижных точек.
При $a=1/2$, из соотношений (\ref{aesoC}) и (\ref{SymmG}) следует$\gamma_{a} = \gamma_{6}-\gamma_{1}=0$ , поэтому $\beta_{a} =-a \gamma_{a}$
 обращается в нуль во всех порядках теории возмущений, независимо от значений параметров $u$ и $w$.
Это означает, что для фиксированных точек (2) и (4), значение $a_{*}=1/2$ на самом деле точное и справедливо во всех порядках разложения по $\varepsilon$ и $\xi$.
Выражение для $w_{*}$ для (3) точки, что соответствует точно решаемой модели Крейчнана, также является точным, в то время как выражения для $u_{*}$ и $w_{*}$
для четвертой точки является лишь членом первого порядка двойного разложения по $\varepsilon$ и $\xi$.
Допустимая неподвижная точка должна быть ИК притягивающей и удовлетворять условиям $u_{*}>0$, $w_{*}>0$ ,
 которые вытекают из физического смысла этих параметров.
В однопетлевом приближении, эти два требования, на самом деле совпадают для всех неподвижных точек.
На рисунке ~\ref{fig:patt} показанны  ИК устойчивые области неподвижных точек (1) - (4) в плоскости $\varepsilon$--$\xi$.
Для точек (1) - (3), матрица (\ref{OmegaDef} либо диагональная, либо треугольная, тогда области устойчивости определяется лишь ее диагональными
элементыми $\Omega_{i} = \partial\beta_{i}/\partial g_{i}$.

\begin{figure}[H]
\begin{center}
%\includegraphics[width=11cm]{PATT.pdf}
\includegraphics[width=11cm]{./grib.jpg}
\caption{\label{fig:patt}
Области устойчивости неподвижных точек в моделе
 (\protect\ref{ActionG}).}
\end{center}
\end{figure}

Для (1) точки имеем \[ \Omega_{u} = -\varepsilon, \quad  \Omega_{w} = -\xi, \quad \Omega_{a} = 0, \]
 так что это допустима область $\varepsilon<0$, $\xi<0$ ( I область на рис. ~\ref{fig:patt} ).
Исчезновение элемента $\Omega_{a}$ отражает тот факт, что параметр $a_{*}$  для I точки произволен.
Для точки (2) можно найти  \[ \Omega_{u} = \varepsilon, \quad
\Omega_{w} = -\xi+u_{*}/8 = -\xi+\varepsilon/12,
\quad \Omega_{a} = u_{*}/4 = \varepsilon/6, \], так что  допустимая область $\varepsilon>0$, $\xi<\varepsilon/12$
(область II на рис.~\ref{fig:patt}).
Для (4) точки матричные  элементы $\partial\beta_{a}/\partial u$,
$\partial\beta_{a}/\partial w$   обращается в нуль,
так что матрица $\Omega$ имеет блочно-треугольный вид, и одно собственное число $\Omega_{a} = u_{*}/4$ легко находится.
Собственные числа оставшейся  $2\times2$  матрицы для $u$, $w$  выглядят довольно сложно, но её определитель  $w_{*}u_{*} (\alpha+5/2)$ простой.
Это выражение показывает, что матрица положительно определена при $u_{*}>0$, $w_{*}>0$ и более тщательный анализ показывает, что это действительно так.
Из явного выражения  (\ref{wu4}) мы заключаем, что для точки (4) допустимая область $\xi>\varepsilon/12$,
$\xi< (3+\alpha) \varepsilon/6$  (область IV на рис.~\ref{fig:patt}).
Оставшийся сектор  III на рис.~\ref{fig:patt} определяется неравенствами
\begin{equation}
\xi > 0, \qquad (3+\alpha) \varepsilon -6\xi<0,
\label{sector3}
\end{equation}
при этом собственные значения матрицы (\ref{OmegaDef}) в  точке (3):
\begin{equation}
\Omega_{u} =  -\varepsilon+2\xi- \frac{4\alpha\xi}{(3+\alpha)}\, f(a_{*}),
\quad \Omega_{w} = \xi, \quad \Omega_{a} = 0,
\label{omga3}
\end{equation}
где $f(a)$ из (\ref{betaG}).
Первое неравенство в (\ref{sector3}) совпадает с $\Omega_{w}>0$, а второе
может быть переписано в виде
\begin{equation}
\Omega_{u}({a_{*}=1/2}) = -\varepsilon+2\xi-
\frac{4\alpha\xi}{(3+\alpha)}\, f(1/2) >0.
\label{omga32}
\end{equation}
Сравнение неравенств (\ref{omga3}) и (\ref{omga32}) показывает, что условие
 $\Omega_{u}>0$  может быть выполнено в тоько области III при условии, что
 $a_{*}$  удовлетворяет неравенству
\begin{equation}
(a_{*}-1/2)^{2} \le \frac{1}{8\alpha\xi}\,
\bigl[ 6\xi- (3+\alpha)\varepsilon \bigr]
\label{kut}
\end{equation}

(Здесь важно, что $f(1/2)=1/2$  является минимальным значением $f$).

Таким образом, мы заключаем, что третий режим, для которого самодействия не имеет значения, лежит в секторе III, гарницы которого описывают неравенства(\ref{sector3}).
Соответствующий аттрактор имеет вид отрезка на прямой  (\ref{line3}), с центром в $a_{*}=1/2$  и определяется неравенством (\ref{kut}).
Этот интервал становится бесконечным при $\alpha=0$ и сужается до одной точки  $a_{*}=1/2$ при $\alpha\to\infty$.
Следует заметить, что для чисто поперечного поля скорости ($\alpha=0$) член с коэффициентом $a_{0}$ в (\ref{nabla}) обнуляется и этот параметр на самом деле исчезает из модели.\\

В однопетлевом приближении (\ref{betaG}), все границы между областями устойчивости - прямые линии. При этом, нет ни пробелов, ни перекрываний между различными областями.
В более высоком порядке, границы между областями II и IV, а также между III и IV могут измениться и стать изогнутыми.
Тем не менее, можно утверждать, что перекрывания областей, а также пустоты между ними не появятся и в более высоких порядках.

В данном случае важно, что модели с $u=0$  или  $w=0$  `` замкнуты относительно перенормировки'', поэтому функции $\beta_{u}$
для  $w=0$ и $\beta_{w}$ для $u=0$  совпадает с  $\beta$  функциями Грибова и Крейчнана во всех порядках теории возмущения.

Остается отметить, что расположение границы между областями III и IV зависит от параметра $\alpha$.
При $\alpha=0$, граница задается лучом $\xi=\varepsilon/2$, что совпадает с результатом, полученным ранее в \cite{AIK} в несжимаемом случае.
Если  $\alpha$ растет, граница начинает вращаться против часовой стрелки и в пределе $\alpha\to\infty$ становится вертикальной линией $\varepsilon=0$, $\xi>0$.

\subsection{Скейлинговые режимы в {\it A} моделе} \label{sec:GPA}


Однопетлевые выражения для $\beta$-функций в моделе (\ref{ActionAR}), (\ref{ActionCR}) легко выводятся из определений
(\ref{betagw}), соотношений  (\ref{aesoC}), (\ref{aesoA}), (\ref{exi}) и явных выражений (\ref{anomA}):

\begin{eqnarray}
\beta_{u} = u \left[ -\varepsilon+ 3u + w(3+\alpha)/2 -w\alpha f(a) \right],
\nonumber \\
\beta_{w} = w \left[ -\xi + w(3+\alpha)/4 \right],
\nonumber \\
\beta_{a} = u(4a-1)/4,
\label{betaA}
\end{eqnarray}
где функция $f(a)=3a^{2}+(a-1)^{2}$ достигает минимального значения
$f(1/4) =3/4$ при $a=1/4$.



(1)  Гауссова  (свободная) линия  неподвижных точек: $u_{*}=w_{*}=0$, $a_{*}- \forall $.
ИК притягивающая при  $\varepsilon<0$, $\xi<0$.\\

(2) Точка  $w_{*}=0$, $u_{*}=\varepsilon/3$, $a_{*}=1/4$, что соответствует читстой модели Грибова
(турбулентный перенос не имеет значения).
ИК притягивающая при $\xi<0$, $\varepsilon<0$.\\

(3) Линия неподвижных точек
\begin{equation}
u_{*}=0, \quad w_{*}=4\xi/(3+\alpha), \quad  a_{*}- \forall,
\label{line4}
\end{equation}
соответствует перемешивающему скалярному полю без самодействия.
Она находится в итервале
\begin{equation}
(a_{*}-1/4)^{2} < \frac{1}{16\alpha\xi} \left[ -\varepsilon(3+\alpha) +
\xi(6-\alpha)  \right],
\label{line5}
\end{equation}
который ИК-притягивающий при $\xi>0$, $\varepsilon< \xi (6-\alpha)/(3+\alpha)$.
Этот интервал становится бесконечным при $\alpha\to0$ и стремится к конечному
 значению $-(\varepsilon+\xi)/16\xi$ при $\alpha\to\infty$  (Обратите внимание, что правая
часть неравенства (\ref{line5}) положительна в ИК области устойчивости).\\

(4) Самая интересная, не тривиальная точка, соответсвует новому режиму (классу универсальности), где важно как взаимодействие, так и перемешивание :
\begin{equation}
w_{*}= 4\xi/(3+\alpha), \quad u_{*} = \frac{ \varepsilon (3+\alpha) -
\xi(6-\alpha)} {3(3+\alpha)}, \quad a_{*}=1/4.
\label{wu44}
\end{equation}
ИК притягивающая при $\xi>0$, $\varepsilon> \xi (6-\alpha)/(3+\alpha)$ \\


С ростом $\alpha$, граница между областями режимов (3) и (4) вращается в
верхней полуплоскости $\varepsilon$--$\xi$ против часовой стрелки от луча  $\xi=\varepsilon/2$ ($\alpha\to0$) к  $\xi=-\varepsilon$ ($\alpha\to\infty$).

ИК- устойчивые области неподвижных точек (1) - (4) в $\varepsilon$--$\xi$-плоскости изображены на рис.~\ref{fig:pattA}.
На этом рисунке показаны области при значении $\alpha<6$,
 когда границы между областями III и IV находится в правом верхнем квадранте, когда же $\alpha>6$ граница попадает в левый верхнмй квадрант.

\begin{figure}[H]
\begin{center}
\includegraphics[width=11cm]{./a.jpg}
\caption{\label{fig:pattA} Области устойчивости неподвижных точек в моделе
(\protect\ref{ActionA}). Приведен случай $\alpha<6$;
для $\alpha>6$ граница между III областями IV лежит в вехнем левом квадранте.}
\end{center}
\end{figure}

\section{Критический скейлинг и критические размерности} \label{sec:DimeNS}

Из существования ИК притягивающих точек уравнений РГ вытекает скейлинговое поведение функций Грина в ИК-области.
При критическом скейлинге фиксируются все ИК-несущественные параметры  ($\lambda$, $\mu$ и константы связи), а
 растягиваются только ИК-существенные (время/частота, координаты/импульсы, $\tau$ поля).

В динамической модели, критические размерности $\Delta_{F}$  ИК существенных величин $F$ задаются соотношениями:
 \begin{eqnarray}
\Delta_{F} = d^{k}_{F}+ \Delta_{\omega} d^{\omega}_{F} + \gamma_{F}^{*},
\qquad  \Delta_{\omega}=2 -\gamma_{\lambda}^{*},
\label{dim}
\end{eqnarray}
с условием нормировки  $\Delta_{k} = 1$; более подробно см., например, \cite{Book3}.
Здесь $d^{k,\omega}_{F}$ -- канонические размерности величин $F$, приведенные в таблице \ref{table1}.
 $\gamma_{F}^{*}$ -- значение соответствующей аномальной размерности (\ref{RGF1}) в фиксированной точке $\gamma_{F}^{*} = \gamma_{F} (u_{*},w_{*},a_{*})$.
В нашем случае:
$\Delta_{\psi} = d/2+ \gamma_{\psi}^{*}$,
$\Delta_{\psi^{\dag}} = d/2+ \gamma_{\psi^{\dag}}^{*}$ для модели Грибова,
$\Delta_{\psi} = d/2-1+ \gamma_{\psi}^{*}$,
$\Delta_{\psi^{\dag}} = d/2+1+ \gamma_{\psi^{\dag}}^{*}$ для модели
{\it A} и $\Delta_{\omega} = 2 + \gamma_{\tau}^{*}$ для обеих моделей.

Подставляя координаты фиксированных точек из разделов ~\ref{sec:GPS}
и~\ref{sec:GPA} в однопетлевые выражения (\ref{anomA}),
(\ref{anomG}) для аномальных размерностей и используя точные соотношения  (\ref{aesoC})--(\ref{aesoG}), получаем в первом порядке выражения для критических размерностей.
Ответы для критических размерностей можно найти в таблицах~\ref{tableG}
и~\ref{tableA} для модели Грибова и модели  {\it A}.


\begin{table}[H]
\caption{Критические размерности полей и параметров в моделе (\protect\ref{ActionG}), (\protect\ref{ActionC}).}
\label{tableG}
\begin{tabular}{|c|c|c|c|c|}
\hline
{} & FP1 & FP2 & FP3 & FP4 \\
\hline
$\Delta_{\omega}$ & 2  & $2-\frac{\varepsilon}{12}$  & $2-\xi$ & $2-\xi$
\\
\hline
$\Delta_{\psi}$ & $d/2$  & $2-\frac{7\varepsilon}{12}$  &
$2-\frac{\varepsilon}{2}+\frac{2\xi\alpha(2a-1)}{(3+\alpha)}$ &
$2- \frac{\varepsilon(18+7\alpha)-6\xi}{6(5+2\alpha)}$ \\
 \hline
$\Delta_{\psi^{\dag}}$ & $d/2$  & $2-\frac{7\varepsilon}{12}$  &
$2-\frac{\varepsilon}{2}-\frac{2\xi\alpha(2a-1)}{(3+\alpha)}$    &
$2- \frac{\varepsilon(18+7\alpha)-6\xi}{6(5+2\alpha)}$ \\
\hline
$\Delta_{\tau}$ & 2  & $2-\frac{\varepsilon}{4}$ & $2-\xi$  &
$2- \frac{\varepsilon(3+\alpha) +3\xi(3+2\alpha)} {3(5+2\alpha)}$ \\
\hline
\end{tabular}
\end{table}

\begin{table}[H]
\caption{Критические размерности полей и параметров в моделе (\protect\ref{ActionA}), (\protect\ref{ActionC}).}
\label{tableA}
\begin{tabular}{|c|c|c|c|c|}
\hline
{} & FP1 & FP2 & FP3 & FP4 \\
\hline
$\Delta_{\omega}$ & 2  & 2  & $2-\xi$ & $2-\xi$ \\
\hline
$\Delta_{\psi}$ & $d/2-1$  & $1-\frac{\varepsilon}{2}$ &
$1+ \frac{\xi-\varepsilon}{2} - \frac{2\xi\alpha(a-1)^{2}} {(3+\alpha)}$ &
$1 - \frac{4\varepsilon(3+\alpha)+\xi(5\alpha-12)}{3(3+\alpha)}$ \\
\hline
$\Delta_{\psi^{\dag}}$ & $d/2+1$  & $3-\frac{\varepsilon}{2}$ &
$3- \frac{\xi+\varepsilon}{2} + \frac{2\xi\alpha(a-1)^{2}} {(3+\alpha)}$ &
$3 - \frac{4\varepsilon(3+\alpha)-\xi(5\alpha-12)}{3(3+\alpha)}$ \\
\hline
$\Delta_{\tau}$ & 2  & $2-\frac{\varepsilon}{3}$  &  $2-\xi$ &
$2- \frac{\varepsilon(3+\alpha)+\xi(3+4\alpha)}{(3+\alpha)}$ \\
\hline
\end{tabular}
\end{table}

Результаты для Гауссовских точек (1) точные. Все результаты для фиксированной точки (2) имеют поправки порядка   $\varepsilon^{2}$ и выше.

Результат $\Delta_{\omega}=2-\xi$  для неподвижных точек (3) и (4) тоже точен. Это следует из общих соотношений  $\gamma_{\lambda}=\gamma_{w}$
 в (\ref{exi}) и $\Delta_{\omega}=2-\gamma_{\lambda}^{*}$ в (\ref{dim}) и тождества $\gamma_{w}^{*}= \xi$, которое
является следствием уравнения $\beta_{w}=0$ для любой неподвижной точки с $w_{*}\ne 0$.
Другие результаты для неподвижной точки (4) имеют высшие поправки по   $\varepsilon$ и $\xi$.
Результат $\Delta_{\tau}=2-\xi$ для неподвижной точки (3) также точен: при  $u=0$, 1-неприводимые функции $\langle \psi^{\dag} \psi \rangle$
имеют только одино-петлевые диаграммы, что приводит только к контрчлену $\psi^{\dag}\partial^{2} \psi$, cр.  \cite{JphysA} для модели Крейчнана.
Отсюда следует, что $Z_{1}=Z_{3}=1$ при $u=0$. Затем из соотношений (\ref{ZZ}) и (\ref{exi}) следует,  $\gamma_{\tau}=-\gamma_{\lambda}=\gamma_{w}$,
что дает$\gamma_{\tau}^{*}= -\xi$ для неподвижных точек с $w_{*}\ne 0$.
Неподвижная точка (3) в таблице~\ref{tableG} иллюстрирует тот факт, что размерности $\Delta_{\psi}$ и $\Delta_{\psi^{\dag}}$ в моделе Грибова меняются
при преобразовании $a\to1-a$.
Остается отметить, что для любой неподвижной точки критическая размерность поля скорости дается точным соотношением  $2\Delta_{v}=-\xi+\Delta_{\omega}$,
которое следует из вида парных корреляционныя функций (\ref{white}), (\ref{Kraich}).

\section{Обсуждение и выводы} \label{sec:Conc}
Мы изучили эффекты влияния турбулентного перемешивания на критическое поведение, при этом уделяли особое внимание {\it сжимаемости} жидкости.
Были рассмотрены два представителя динамических моделей критического поведения: модель {\it A}, описывающая релаксационную динамику без
сохранения параметра порядка в критической равновесной системе, и сильно неравновесная модель Грибова, которая описывает распространение процессов в системе реакция-диффузия.
Турбулентное перемешивание моделировалось моделью Казанцева - Крейчнана с Гауссовым полем скорости с степенным спектром $\propto k^{-d-\xi}$.
Обе стохастические задачи можно переформулировать в виде мультипликативно перенормируемых  теоретико полевых моделей, что позволяет применять метод РГ для анализа их ИК поведения.
Было показанно, что, в зависимости от соотношения между пространственной размерностью  $d$ и показателем  $\xi$, обе модели демонстрируют четыре различных вида критического поведения
, связанных с четырьмя возможными неподвижными точками уравнения РГ.
Три неподвижные точки соответствуют известным режимам: (1) Гауссовой неподвижной точке; (2) критическому поведению типичному для чистой модели без турбулентного переноса
(то есть, модель {\it A} или Грибова); (3) скалярному полю без самодействия (нелинейность параметра порядка в исходных динамических
уравнениях является несущественной).
Наиболее интересной четвертый точке соответствует новый тип критического поведения (4), в котором важны как нелинейность, так и
турбулентное перемешивание. Критические показатели зависят от $d$, $\xi$ и параметра сжимаемости $\alpha$.
Были вычисленны критические индексы и области устойчивости для всех режимов в однопетлевом приближении,
что соответствует главным членам двойного разложения по параметрам $\xi$ и $\varepsilon=4-d$.
Было показанно, что для обеих моделей, сжимаемость усиливает роль нелинейных членов в динамических уравнениях.
В плоскости $\varepsilon$--$\xi$, область устойчивости (4) режима становится шире при возрастании степени сжимаемости.

Проиллюстрируем эти общие слова на примере облака частиц в системе реакция-диффузия, распространяющегося в близкой к критической турбулентной среде.
Среднеквадратичный радиус $R(t)$ облака частиц, связан с  функцией отклика (\ref{respd}) в координатно-временном представлении следующим образом:
\begin{eqnarray}
R^{2}(t) = \int d{\bf x}\ x^{2}\, G(t,{\bf x}), \quad
G(t,{\bf x}) = \langle \psi (t,{\bf x}) \psi^{\dag} (0,{\bf 0}) \rangle,
\quad x=|{\bf x}|.
\label{Rad}
\end{eqnarray}
Для  функции $G(t,{\bf x})$ скейлинговые соотношения в предыдущем разделе дают следующие ИК-асимптотики:
\begin{eqnarray}
G(t,{\bf x}) = x^{-\Delta_{\psi}-\Delta_{\psi^{\dag}}} \, F
\left(\, \frac{x} { t^ {1/\Delta_{\omega}} }, \,
\frac{\tau}{t^{\Delta_{\tau}/\Delta_{\omega}}}  \right),
\label{Green}
\end{eqnarray}
с некоторой функцией $F$.
Подставляя (\ref{Green}) в (\ref{Rad}) получаем скейлинговое выражение радиуса:
\begin{eqnarray}
R^2(t) = t^{ (d+2 -\Delta_{\psi}-\Delta_{\psi^{\dag}})/\Delta_{\omega} }
\, f \left( \frac{\tau}{t^{\Delta_{\tau}/\Delta_{\omega}}}  \right),
\label{R3}
\end{eqnarray}
где скейлинговая функция $f$  связана с $ F $ из (\ref{Green})
\[ f (z) = \int d{\bf x}\, x^{2-2\Delta_{\psi}} \, F(x,z). \]
Непосредственно в критической точке (предполагается, что функции  $f$  конечна при $\tau=0$) получаем из
(\ref{R3}) степенной закон для радиуса:
\begin{eqnarray}
R^2(t) \propto t^\Omega, \quad \Omega \equiv { (d+2
-\Delta_{\psi}-\Delta_{\psi^{\dag}})/ \Delta_{\omega} } =
{(2-\gamma_{\psi}^{*}-\gamma_{\psi^{\dag}}^{*})/\Delta_{\omega} };
\label{R4}
\end{eqnarray}
Последнее равенство следует из соотношения  \ref{dim}.
Для Гауссовой неподвижной точки имеем обычный закон диффузии  $R(t)\propto t^{1/2}$.
Для режима (3) получаем результат $R(t)\propto t^{1/(2-\xi)}$. Для Колмогоровского
значения $\xi=4/3$ , $R(t)\propto t^{3/2}$ он находится в соответствии с ``законом Ричардсона 4/3''
 $dR^{2}/dt \propto R^{4/3}$ для турбулентности.
Для двух других неподвижных точек показатели в (\ref{R3}), (\ref{R4}) задаются бесконечными рядами
по $\varepsilon$  (для точки 2) и  $\varepsilon$, $\xi$(для точки 4). Ответы в первом приближении легко получить из результатов, приведенных в таблицах~\ref{tableG} и~\ref{tableA}.
В случае несжимаемой жидкости  ($\alpha=0$), наиболее реалистичные значения
 $d=2$ или 3  и $\xi=4/3$ лежат в области режима (3), так что
 распространение облака полностью определяется турбулентным переносом и описывается степенным законом (\ref{R4}) с точным показателем  $\Omega^{(3)}= 2/(2-\xi)$.
При возрастании степени сжимаемости $\alpha$, область устойчивости (4) режима становится шире (см. обсуждение в разделе~\ref{sec:FPS}).
Когда $\alpha$ cтановится достаточно большим, физические величины  $d$ и $\xi$  попадают в область устойчивости нового режим (4), происходит изменение критического поведения.
Новый показатель в (\ref{R4}) можно представить в виде
\begin{eqnarray}
\Omega^{(4)}= \Omega^{(3)} + \delta\Omega, \quad
\delta\Omega = - (\gamma_{\psi}^{*}+\gamma_{\psi^{\dag}}^{*}) / (2-\xi)
\label{R5}
\end{eqnarray}
(Напомним, что  $\Delta_{\omega}=(2-\xi)$ точное для обоих режимов (3) и (4)).
В моделе Грибова, подстановка однопетлевых выражений для размерностей $\gamma_{\psi}^{*}$ и $\gamma_{\psi^{\dag}}^{*}$ из таблицы ~\ref{tableG} в (\ref{R5}) дает
\begin{eqnarray}
\delta\Omega = \frac{(3+\alpha)\varepsilon-6\xi}{3(5+2\alpha)(2-\xi)}.
\label{R6}
\end{eqnarray}

Простой анализ выражения (\ref{R6}) показывает, что в области устойчивости (4) режима, величина $\delta\Omega$ положительна и монотонно растет с $\alpha$.
В самом деле, непосредственно проверяется, что отношения $\delta\Omega>0$
и $\partial_{\alpha}\delta\Omega>0$ эквивалентны неравенствам $(3+\alpha)\varepsilon-6\xi>0$ и $-\varepsilon+12\xi>0$,
которые определяют область устойчивости режима (\ref{wu4}) в однопетлевом приближении.
Таким образом, распространение облака становится быстрее в сравнении с чистым турбулентным переносом,
за счет комбинированного воздействия перемешивания и нелинейных членов, и ускоряется при росте степени сжимаемости.
Для модели {\it A}, показателей $\Omega$ в (\ref{R4}) для режимов (3) и (4) совпадают в однопетлевом приближении, но основной количественный вывод остается прежним: 
сжимаемости повышает роль нелинейных членов и ведет к расширению области устойчивости (4) критическом режиме.
Хотя системы реакции-диффузия, подвергающиеся турбулентному перемешиванию широко распространены в природе, на данный момент мы не можем предложить любые существующие или
возможные экспериментальные тесты, которые могут быть использованы для проверки аналитических результатов этой работы.
Проблема заключается в тщательном измерении показателей в таких неидеальных системах, при этом особое внимание должно уделяться управлению степени сжимаемости.
Скорее всего, развитие в этой области будут связаны с численным моделированием, как это произошло для систем реакции-диффузии без перемешивания \cite{Hinr}.
\newpage


\section*{References}
\begin{thebibliography}{99}

\bibitem{Zinn} Zinn-Justin J 1989 {\it Quantum Field Theory and Critical
Phenomena} (Oxford: Clarendon)

\bibitem{Book3} Vasil'ev A N 2004 {\it The Field Theoretic Renormalization
Group in Critical Behavior Theory and Stochastic Dynamics}
(Boca Raton: Chapman \& Hall/CRC)

\bibitem{HH} Halperin B I and Hohenberg P C 1977 {\it Rev. Mod. Phys.}
{\bf 49} 435; Folk R and Moser G 2006 {\it J. Phys. A: Math. Gen.}
{\bf 39} R207

\bibitem{Hinr} Hinrichsen H 2000 {\it Adv. Phys.} {\bf 49} 815;
 \'Odor G 2004 {\it Rev. Mod. Phys.} {\bf 76} 663

\bibitem{JT} Janssen H-K and T\"{a}uber U C 2004 {\it Ann. Phys. (NY)}
{\bf 315} 147

\bibitem{Ivanov} Ivanov D Yu 2003 {\it Critical Behaviour of
Non-Idealized Systems} (Moscow: Fizmatlit) [in Russian]

\bibitem{quench} Khmel'nitski D E 1975 {\it Sov. Phys. JETP} {\bf 41} 981;
Shalaev B N 1977 {\it Sov. Phys. JETP} {\bf 26} 1204;
Janssen H-K, Oerding K and Sengespeick E 1995
{\it J. Phys. A: Math. Gen.} {\bf 28} 6073

\bibitem{Satten} Satten G and Ronis D 1985 {\it Phys. Rev. Lett.}
{\bf 55} 91; 1986 {\it Phys. Rev.} A {\bf 33} 3415

\bibitem{Onuki} Onuki A and Kawasaki K 1980 {\it Progr. Theor. Phys.}
{\bf 63} 122;
Onuki A, Yamazaki K and Kawasaki K 1981 {\it Ann. Phys.} {\bf 131} 217;
Imaeda T, Onuki A and Kawasaki K 1984 {\it Progr. Theor. Phys.} {\bf 71} 16

\bibitem{Beysens} Beysens D, Gbadamassi M and Boyer L 1979
{\it Phys. Rev. Lett} {\bf 43} 1253;
Beysens D and Gbadamassi M 1979 {\it J. Phys. Lett.} {\bf 40} L565

\bibitem{Ruiz} Ruiz R and Nelson D R 1981 {\it Phys. Rev.} A {\bf 23}
3224; {\bf 24} 2727;
Aronowitz A and Nelson D R 1984 {\it Phys. Rev.} A {\bf 29} 2012

\bibitem{AHH} Antonov N V, Hnatich M and Honkonen J 2006
{\it J. Phys. A: Math. Gen.} {\bf 39} 7867

\bibitem{Alexa} Antonov N V and Ignatieva A A 2006
{\it J. Phys. A: Math. Gen.} {\bf 39} 13593

\bibitem{AIK} Antonov N V, Iglovikov V I and Kapustin A S 2009
{\it J. Phys. A: Math. Theor.} {\bf 42} 135001

\bibitem{AIM} Antonov N V, Ignatieva A A and Malyshev A V 2010
E-print LANL arXiv:1003.2855 [cond-mat]; to appear in {\it PEPAN
(Phys. Elementary Particles and Atomic Nuclei, published by JINR)} {\bf 41}

\bibitem{FGV} Falkovich G, Gaw\c{e}dzki K and Vergassola M  2001
{\it Rev. Mod. Phys.} {\bf 73} 913

\bibitem{JphysA} Antonov N V 2006 {\it J. Phys. A: Math. Gen.} {\bf 39} 7825

\bibitem{Compress} Antonov N V, Hnatich M and Nalimov M Yu 1999
{\it Phys. Rev.} E {\bf 60} 4043

\bibitem{vanK} van Kampen N G 2007 {\it Stochastic Processes in Physics
and Chemistry, 3rd ed.} (Amsterdam: North Holland)

\bibitem{Sak} Sak J 1973 {\it Phys. Rev.} B {\bf 8} 281;
Honkonen J and Nalimov M Yu 1989 {\it J. Phys. A: Math. Gen.} {\bf 22} 751
Janssen H-K 1998 {\it Phys. Rev.} E {\bf 58} R2673;
Antonov N V 1999 {\it Phys. Rev.} E {\bf 60} 6691;
2000 {\it Physica} D {\bf 144} 370

\bibitem{Levy} Janssen H-K, Oerding K, van Wijland F and Hilhorst H J
1999 {\it Eur. Phys. J.} B {\bf 7} 137; Janssen H-K and Stenull O 2008
{\it Phys. Rev.} E {\bf 78} 061117

\end{thebibliography}
\end{document}







Вопросы:
1)Как я понтмаю нелинейность- это взаимодействие в действии или добавка в динамических уравнениях. Каким образом можео понять что
существование сжимипмости у перемешивания повышает роль нелинейных
членов в динамических уравнениях?




\end{document}
