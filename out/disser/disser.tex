%% Преамбула TeX-файла диссертации.
\documentclass []{rusthesis} % Стиль (по умолчанию 14pt)
\usepackage[utf8x]{inputenc}
\usepackage{ucs}
\usepackage{amsmath} 
\usepackage{mathtext}
\usepackage{amsfonts}
\usepackage{upgreek}
\usepackage[english,russian]{babel}
\usepackage{graphicx}
\usepackage{textcomp}
\usepackage{tikz}
\usepackage{ccaption}

%%%%%% DEFINITIONS %%%%%%%%%%%%%%%%%%%%%%%%%%%
\def\eps{\varepsilon}
\def\Dm{\widetilde{\cal D}_{\mu}}
\def\const{{\rm const\,}}
\def\D{{\mathcal D}}
\def\L{{\mathcal L}}
\def\A{{\mathcal A}}
\def\S{{\mathcal S}}
\def\E{{\mathcal E}}
\def\x{{\bf x}}
\def\p{{\bf p}}
\def\k{{\bf k}}
\def\q{{\bf q}}
\def\bfr{{\bf r}}
\def\h{{\bf h}}
\def\n{{\bf n}}
\def\bfx{{\bf x}}
\def\bfv{{\bf v}}
\def\bfu{{\bf u}}

\usepackage{graphicx}	% Пакет для включения рисунков
\usepackage{amssymb}    %специальные математические символы
\usepackage{cite}       %правильное цитирование
\usepackage{textcomp}   %специальные текстовые символы
\usepackage{mathtext}
\setcounter{tocdepth}{3}
\setcounter{secnumdepth}{3}


\def\printcitestart{[}	% левая и правая скобки для ссылки на
\def\printcitefinish{]}	% литературу по умолчанию будут [ ]

%\udk{}
\institut{САНКТ-ПЕТЕРБУРГСКИЙ ГОСУДАРСТВЕННЫЙ УНИВЕРСИТЕТ}
\minister{}
\title{ВЛИЯНИЕ ТУРБУЛЕНТНОГО ПЕРЕМЕШИВАНИЯ НА КРИТИЧЕСКОЕ ПОВЕДЕНИЕ В ПРИСУТСТВИИ СЖИМАЕМОСТИ}
\author{Капустин Александр Сергеевич}
\field{01.04.02 --- теоретическая физика}
\degree{кандидата физико-математических}
\chiefdegree{профессор физико-математических}
\chief{Антонов~Н.~В.}
\othermembers{}
\numberofmembers{1}
\degreecity{Санкт-Петербург}
\degreeyear{2014}
\date{}

\napravax{На правах рукописи}
\dnsus{Диссертация на соискание учёной степени}
\nauk{наук}
\nauruk{Научный руководитель}

\sloppy			% Не выходить за границы бокса
\Russian






\begin{document}
\maketitle

%%%%%%%%%%%%%%%%%%%%%%%%%%%%%%%%%%%%%%%%%%%%%%%%%%%%%%%%%%%%%%
% Введение                                                   %
%%%%%%%%%%%%%%%%%%%%%%%%%%%%%%%%%%%%%%%%%%%%%%%%%%%%%%%%%%%%%%

\chapter*{Введение}			% * - чтобы не нумеровалось
\addcontentsline{toc}{chapter}{Введение}	% Но тогда надо в содержание...
\markboth{}{}				% и в клонтитулы (myheadings)

Многочисленные физические системы самой разной природы (магнетики, системы типа жидкость-пар, бинарные смеси) обнаруживают необычное поведение вблизи точки фазового перехода второго рода (критической точки). Корреляционные функции и некотрые термодинамические величины (теплоемкость, восприимчивость и др.) демонстрируют степенное поведение (критический скейлинг). При этом соответствующие критические показатели оказываются универсальными, т.е. зависящими лишь от глобальных характеристик системы, таких как размерность пространства или симметрия. Это свойство позволяет объединить системы, различающихся множеством несущественных (с точки зрения критического поведения) деталей в единые классы универсальности, характеризуемые одним и тем же набором критических показателей и скейлинговых функций.

Последовательное количественное описание критического поведения можно дать в рамках метода РГ. В ренормгрупповом подходе возможные критические режимы связыаются с инфракрасно-устойчивыми  неподвижными точками некотрой ренормируемой теоретико-полевой модели. Цель теории - выделить основные классы универсальности, установить существование нужных неподвижных точек для соответствующих моделей, вычислить для них критические размерности и скейлинговые функции в рамках какой-либо последовательной теории возмущений. Подробное изложение ренормгрупповых методов и их применения к теории фазовых переходов можно найти в монографии \cite{Zinn}.

Наиболее типичные фазовые переходы в равновесных системах принадлежат к классу универсальности $O(N)$-симметричной модели $\phi^4 N$- компонентного скалярного параметра порядка. Её критические показатели зависят только от $N$ и размерности пространства $d$. Они могут быть рассчитаны в виде разложения по $\varepsilon = 4-d$ или в рамках других схем теории возмущения, см. монографии \cite{Zinn, Book3} и цитированную там литературу. Большой интерес в последние годы привлекают фазовые переходы в системах, далеких от состояния термодинамического равновесия. Их критическое поведение гораздо более многообразно и пока недостаточно хорошо изучено. Показательным примером являются разнообразные и часто встречающиеся в природе процессы распространения, такие как: эпидемии, каталитические реакции, лесные пожары, диффузия в пористых или флуктуирующих средах. Для определенности мы будем использовать терминологию первого случая (эпидемий). В зависимости от конкретных условий процесс распространения (в случае эпидемий -- распространения инфекции) может либо продолжаться и охватить всю популяцию, либо полностью прекратиться через некотрое время. Как теперь известно, переход (при изменении параметров типа вероятностей инфицирования) от флуктурирующего (активного) к абсорбционному (неактивному) состоянию является фазовым переходом второго рода и описывается несколькими классами универсальности. Простая модель описывающая распространения агента (например, инфекционные заболевания) - модель Грибова. Эта модель эквивалентна Реджеонной теории поля и была изучена в рамках РГ подхода и $\varepsilon-$ разложения. Однако, как уже давно поняли, поведение реальной системы вблизи критической точки является чрезвычайно чувствительным к внешним воздействиям: к гравитации, движению самой среды, наличию примесей и т.д. Более того, некотрые нарушения (примеси или турбулентное перемешивание) могут привести к совершенно новым типам критического поведения с богатыми и весьма экзотическими свойствами, например, разложение по $\sqrt{\varepsilon}$, а не по $\varepsilon$ \cite{quench, Satten}. Эти вопросы становятся особенно актуальными для неравновесных фазовых переходов, так как идеальные условия для <<чистого>> стационарного критического состояния вряд ли могут быть достигнуты в реальных химических или биологических системах, а также влияние различных нарушений (гравитации, перемешивание и т.д.) никогда не может быть полностью исключено. В частности, нельзя пренебрегать эффектом турбулентности в химических каталитических реакциях или лесных пожарах. Так же можно предположить, что атмосферная турбулентность может сыграть важную роль в распространении инфекционных заболеваний на летающих насекомых или птицах. Исследование влияния различных видов перемешиваний (ламинарных сдвиговых течений, турбулентной конвекции и так далее) на поведение критической жидкости (например, бинарная жидкая смесь) показало, что это перемешивание может нарушить обычное критическое поведение системы, характерное для $\phi^4$ или Реджеонной модели. В данной работе мы изучаем влияние эффектов турбулентного перемешивания на критическое поведение трех систем вблизи критических точек, обращая особое внимание на сжимаемость жидкости. Для описания фазовых переходов мы используем три модели динамического критического поведения. Первая из моделей описывает фазовый переход между т.н. абсорбционным (неактивным) и флуктуацинным (активным) состояниями некотрой реакционно-диффузионной системы. Более конкретно будет рассмотрен процесс типа Грибова, называемый так потому, что соответствующая полевая модель близка к хорошо известной Реджеонной теории поля \cite{Zinn}. Вторая же - релаксационная динамика несохраняющегося скалярного параметра порядка для модели с взаимодействием типа $\phi^4$ (в работе будет представлена как модель А).

Третья модель - модель Ашкина-Теллера-Поттса (или просто Поттсa) - целый класс моделей, см. \cite{Ashkin, Baxter, Golner, Zia, Priest, Amit, Alcantara}. Такая модель описывает некоторую систему, которая локально имеет $n$ состояний, при этом энергия любой заданной конфигурации зависит от того, находятся ли элементы в соседних узлах в том же состоянии или нет. В непрерывной формулировке модель Поттса удобно представить эффективным гамильнонианом для $n$-компонентного параметра порядка с трилинейным членом взаимодействия и инвариантным относительно группы симметрии $n$-мерного гипертетраэдра см. \cite{Golner, Zia, Priest, Amit, Alcantara}.

Модели типа Поттса имеют многочисленные физические приложения: твердые и магнитные материалы с нетривиальной симметрией, спиновые стекла, переходы от нематического к изотропному состоянию в жидких кристаллах, проблемы перколяции, рандомные резисторные цепи и многие другие, см. работы \cite{Golner, Zia, Priest, Amit, Alcantara} и ссылки в них. 

Данный класс моделей (особенно случай $n=d=2$ - знаменитая двумерная модель Изинга) давно стал источником вдохновений для новых физических и математических идей, таких как интегрируемость, конформная инвариантность, дискретная голоморфность и т.д. см. \cite{Cardy, IMConform}  и ссылки в них.

Вопрос о природе фазового перехода в модели Поттса имеет долгую и запутанную историю, см. обсуждение в \cite{Amit}. Согласно теории критического поведения Ландау в модели типа Поттса не может быть фазового перехода второго рода из-за трехлинейной вершины взаимодействия в гаммильтониане. С другой стороны, точные двумерные результаты, численное моделирование и ренормгрупповой анализ предполагают, для достаточно малого $n$, фазовый переход второго рода в модели Поттса существует. Мы не будем пытаться пролить новый свет на этот интересный и сложный вопрос: для начала, он должен быть решен для исходной статической проблемы. В данной работе мы принимаем точку зрения, что существование ИК-притягивающей неподвижной точки уравнения РГ подразумевает существование некоторого ИК асимптотического режима и, следовательно, существование какого-то критического сосотяния.



Для модели турбулентного перемешивания мы используем известную модель Казанцева - Обухова - Крейчнана, которая верно передает основные черты реальной развитой турбулентности. В последнее время эта модель приобрела огромную популярность в связи с проблемой аномального скейлинга в задачах турбулентности и турбулентного переноса пассивной примеси, см. обзор \cite{JT} и цитированную там литературу. В нашем случае особенно важно, что такое описание перемешивания позволяет легко моделировать сжимаемую жидкость, что оказывается весьма трудно, если скорость моделируется динамическими уравнениями.


План работы таков. В разделе 2 дано описание моделей в терминах теории поля с соответствующей диаграмной техникой. В разделе 3 анализируются ультрафиолетовые расходимости моделей. Показано, что модели являются мультипликативно-ренормируемыми, и преведены ренормированные функционалы действия. Таким образом, можно получить уравнения РГ и ввести РГ-функции ($\beta$ функции и аномальные размерности $\gamma$) стандартным образом (раздел 4). В разделе 5 показано

%%TODO: дописать в соответсвии с dip.pdf 3 p.
%%
%%
%%
%%
%%
%%\textbf{Актуальность темы.}
%%TODO
%%
%%\textbf{Целью} настоящей диссертационной работы является изучение магнитных свойств инвертированных опалоподобных наноструктур на основе никеля и кобальта, а также определение зависимости этих свойств от пространственной структуры ИОПС, типа материала-заполнителя и влияния внешнего магнитного поля.
%%
%%В соответствии с целью исследования были поставлены следующие основные задачи:
%%\begin{enumerate}
%%	\item Аттестовать структуру инвертированных опалоподобных кристаллов на основе никеля и кобальта методами сканирующей электронной микроскопии и ультрамалоугловой дифракции синхротронного излучения.
%%	\item Аттестовать фазовый состав и структуру материалов-заполнителей методом широкоугольной дифракции синхротронного излучения.
%%	\item Изучить влияние магнитожёсткости/магнитомягкости материала-заполнителя, пространственной анизотропии инвертированной опалоподобной структуры и двумерной анизотропии плёнки на магнитные свойства образцов инвертированных опалоподобных кристаллов методом магнитометрии с применением сверхпроводящего квантового интерференционного магнетометра (СКВИД-магнитометрии).
%%	\item Исследовать зависимость поведения средней намагниченности инвертированных опалоподобных структур на основе никеля и кобальта от магнитного поля при различных углах между направлением вектора напряжённости внешнего магнитного поля и плоскостью образца для образцов различной толщины. Основываясь на полученных результатах, определить механизмы перемагничивания, реализующиеся в таких объектах.
%%	\item Провести исследование поведения магнитной структуры во внешнем магнитном поле в ферромагнитных инвертированных опалоподобных наноструктурах на основе никеля и кобальта методом малоугловой дифракции поляризованных нейтронов.
%%	\item На основе полученных данных построить модель распределения намагниченности внутри никелевых и кобальтовых инвертированных опалоподобных структур при различных значениях величины и направления вектора напряжённости внешнего магнитного поля.
%%\end{enumerate}
%%
%%\textbf{Научная новизна.}
%%TODO
%%
%%\textbf{Научная и практическая ценность.}
%%TODO
%%
%%\textbf{Основные положения, выносимые на защиту:}
%%TODO
%%
%%\underline{Публикации}. По теме диссертации опубликовано 10 научных работ в изданиях, рекомендованных ВАК РФ.
%%TODO
%%\begin{enumerate}
%%\item S.V.~Grigoriev, K.S.~Napolskii, N.A.~Grigoryeva, A.V.~Vasilieva, A.A.~Mistonov, A.S.~Sinitskii, H.~Eckerlebe, D.Yu.~Chernyshov, A.V.~Petukhov, D.V.~Belov, A.A.~Eliseev, A.V.~Lukashin, Yu.~D.~Tretyakov, Phys. Rev. В, 79, 2009, 045123
%%\item В.В.~Абрамова, А.С.~Синицкий, Н.А.~Григорьева, С.В.~Григорьев, Д.В.~Белов, А.В.~Петухов, А.А.~Мистонов, А.В.~Васильева, Ю.Д.~Третьяков, ЖЭТФ, т. 136, вып.1(7), с.1-7, 2009
%%riev, S.N.~Samarin, Phys.Rev.B v. 86, 184431(2012)
%%
%%\end{enumerate}
%%
%%\underline{Структура и объем работы}. Диссертация состоит из введения, 6 глав, заключения и списка литературы из 129 наименований. Работа изложена на 174 страницах и содержит 51 рисунок и 1 таблицу.
%%

%%%%%%%%%%%%%%%%%%%%%%%%%%%%%%%%%%%%%%%%%%%%%%%%%%%%%%%%%%%%%%
% Глава 1                                                   %
%%%%%%%%%%%%%%%%%%%%%%%%%%%%%%%%%%%%%%%%%%%%%%%%%%%%%%%%%%%%%%

% Глава 1.

\chapter{Описание моделей. Теоретико-полевая формулировка моделей}
\label{sec:QFT}

В Ланжевеновой формулировке формулировке наши модели определены стохастическими  дифференциальными уравнениями для параметра порядка $\psi = \psi(x)$ c $x \equiv {t,\bf x}$:


%% TODO: верт черта в формуле
\begin{equation}
\partial_t\psi(x) =-\lambda_0 \Bigr.\frac{\delta H(\psi)}{\delta \psi (\bfx)}\Bigr|_{\x \rightarrow x} + \zeta(x),
\label{eq:Langeven}	
\end{equation}


В случае модели Поттса мы имеем дело с $n$- компонентным параметром порядка $\psi_a = \psi(t,{\bf x})_a$, и Ланжевеново описание очевидным
образом обобщается:

%% TODO: верт черта в формуле
\begin{equation}
\partial_t\psi(x)_a =-\lambda_0 \Bigr.\frac{\delta H(\psi)}{\delta \psi_a (\bfx)}\Bigr|_{\x \rightarrow x} + \zeta(x)_a,
\label{eq:Langeven_potts}	
\end{equation}

Где $\partial_t=\partial/\partial t$, $\lambda_0>0$ - кинематический коэффициент. Гауссов шум $\zeta=\zeta(t,x)$
с нулевым средним задается парной корреляционной функцией:

\begin{eqnarray}
\langle \zeta (t,{\bf x})\zeta (t',{\bf x'}) \rangle =
2 \lambda_{0}  \delta(t-t')\delta^{(d)}({\bf x}-{\bf x}')
\label{eq:zeta_A}
\end{eqnarray}
Для модели {\it A},
\begin{equation}
\langle \zeta (t,{\bf x})\zeta (t',{\bf x'}) \rangle = g_{0}\lambda_{0}\,
\psi(t,{\bf x})\,   \delta(t-t')\delta^{(d)}({\bf x}-{\bf x}')
\label{eq:zeta_G}
\end{equation}
Для процесса Грибова,
\begin{equation}
\langle \zeta_a (t,{\bf x})\zeta_b (t',{\bf x'}) \rangle = 2\lambda_{0}\,
\delta_{ab} \delta(t-t')\delta^{(d)}({\bf x}-{\bf x}')
\label{eq:zeta_P}
\end{equation}
Для модели Поттса.

$d$ - размерность пространства ${\bf x}$. Множитель $\psi$ в передней части коррелятора
(\ref{eq:zeta_G}) обеспечивает полное исчезновение флуктуаций в неактивной (абсорбционной) фазе. (
{\itТакая специфическая форма амплитуды может быть получена в результате перерастяжения шума и поля}).
%% TODO В случае Поттса такое требование - чтоб одновременные корреляционные функции стохастической проблемы задавались с весо exp(-H)
А член $2\lambda_0$ в (\ref{eq:zeta_A}, \ref{eq:zeta_P}) связан с флуктуационно-диссипационной тоеремой: он обеспечивает
соответствие с статистической моделью $\psi^4$. Здесь и далее затравочные (неренормированные) параметры
обозначаются индексом "0". Их ренормированные аналоги (без индекса) появятся позже.

Гамильтониан в уравнении (\ref{eq:Langeven}) задается следующими соотношениями:
\begin{equation}
H(\psi) = \int d{\bf x} \left \{ -\frac{1}{2} \psi({\bf x}) \partial^2 \psi({\bf x}) \, 
+\frac{\tau_0}{2} \psi({\bf x}) \psi({\bf x}) + V(\psi)\right \}
\label{eq:H_A_G}
\end{equation}

Нелинейный член $V(\psi)$ для модели ${\it A}$ имеет вид $V(\psi)=u_0 \psi^4/4!$, для процесса Грибова $V(\psi)=g_0 \psi^3/3!$; 
$g_0$ и $u_0 >0 $ - константы связи. Вблизи критической точки статический Гамильтониан $H(\psi)$ для модели Поттса представим в виде \cite{Golner, Zia, Priest}:

\begin{eqnarray}
H(\psi) = \int d{\bf x} \{ -\frac{1}{2} \psi_a({\bf x}) \partial^2 \psi_a({\bf x}) \, 
+\frac{\tau_0}{2} \psi_a({\bf x}) \psi_a({\bf x}) \nonumber \\
+ \frac{g_0}{3!}R_{abc} \psi_a({\bf x}) \psi_b({\bf x}) \psi_c({\bf x})  \},
\label{eq:H_P}
\end{eqnarray}

где $\partial_i = \partial/\partial x_i$ - частная производная, $\partial^2 = \partial_i \partial_i$ - оператор
Лапласа. $\tau_0 \propto (T-T_c)$ - отклонение температуры или ее аналога от критического значения, $g_0>g$- константа связи. 
Во всех наших формулах мы подразумеваем суммирование по повторяющимся индексам ($a,b,c = 1, ... ,n$ и $i =1, ... ,d$). 
После взятия функциональной производной $\frac{\delta H(\psi)}{\delta \psi_a (\bfx)}$ в формулах (\ref{eq:Langeven}, \ref{eq:Langeven_potts}) поле $\psi({\bf x})$ необходимо заменить на $\psi(x) = \psi({\bf x}, t) $

Следуя \cite{Alcantara}, мы будем рассматривать общий случай c определенной группой симметрии $G$, для которой
неприводимый инвариант - тензор третьего ранга $R_{abc}$, который без ограничения общности можно считать симметричным. В однопетлевом приближении нам достаточно знать лишь коэффициенты $R_{1,2}$, определяемые из соотношений:

\begin{eqnarray}
R_{abc}R_{abe} = R_1 \delta_{ce}, \quad R_{aec}R_{chb}R_{bfa} = R_2 R_{ehf}.
\label{eq:R12}
\end{eqnarray}

В оригинальной модели Поттса $G=Z_n$ - группа симметрии гипертетраэдра в $n$ - мерном пространстве. В
данном случае тензор $R_{abc}$ легко представим в терминах набора $(n+1)$ вектора $e^{\alpha}$, определенных в \cite{Golner, Zia}:
\begin{eqnarray}
R_{abc} = \sum_{\alpha} e^{\alpha}_{a}e^{\alpha}_{b}e^{\alpha}_{c},
\label{eq:R}
\end{eqnarray}
где $e^{\alpha}_{a}$ удовлетворяют соотношениям:

\begin{eqnarray}
\sum_{\alpha=1}^{n+1} e^{\alpha}_{a} = 0,\quad
\sum_{\alpha=1}^{n+1} e^{\alpha}_{a}e^{\alpha}_{b} = (n+1)\delta_{ab}, \quad
\sum_{a=1}^{n} e^{\alpha}_{a}e^{\beta}_{a} = (n+1)\delta^{\alpha \beta} -1.
\label{eq:e_a_alpha}
\end{eqnarray}

Соотношения (\ref{eq:e_a_alpha}) позволяют представить коэффициены (\ref{eq:R12}) в следующем виде:
\begin{eqnarray}
R_1 = (n+1)^2(n-1), \quad
R_2 = (n+1)^2(n-2)
\label{eq:R12_n}
\end{eqnarray}

Стохастические проблемы (\ref{eq:Langeven}, \ref{eq:Langeven_potts}, \ref{eq:zeta_A}, \ref{eq:zeta_G}, \ref{eq:zeta_P}) 
могут быть переформулированы в виде теоретико-полевых моделей с удвоенным набором полей $\Phi={\psi, \psi^\dagger}$. В
такой формулировке они будут описываться функционалами действия:

\begin{eqnarray}
\S(\psi,\psi^{\dagger}) =  \psi^{\dagger}
\left(-\partial_{t}+\lambda_{0} \partial^{2}- \lambda_{0}\tau_{0}\right)
\psi + \lambda_{0} (\psi^{\dagger})^2 - u_{0} \psi^{\dagger}\psi^3/3!
\label{actionA}
\end{eqnarray}
для модели {\it A},
\begin{eqnarray}
\S(\psi,\psi^{\dag}) =  \psi^{\dag}
(-\partial_{t}+\lambda_{0} \partial^{2}- \lambda_{0}\tau_{0}) \psi
+ \frac{g_{0}\lambda_{0}}{2} \left\{ (\psi^{\dagger})^2\psi -
\psi^{\dagger}\psi^2  \right\}
\label{actionG}
\end{eqnarray}
для процесса Грибова,
\begin{eqnarray}
\S(\psi,\psi^{\dagger}) =  \psi^{\dagger}_a
\left(-\partial_{t}+\lambda_{0} \partial^{2}- \lambda_{0}\tau_{0}\right)
\psi_a + \lambda_{0} \psi_a^{\dagger} \psi_a^{\dagger}-\,
\frac{g_{0}\lambda_0}{2} R_{abc}\psi_a^{\dagger}\psi_b\psi_c
\label{actionP}
\end{eqnarray}
Для модели Поттса. Где $\psi^{\dag}=\psi^{\dag}(t,{\bf x})$ - вспомогательное поле "поле отклика".
В приведенных формулах подразумевается интегрирование по аргументам полей и суммирование по индексам, например:

\[  \psi^{\dag}\partial_{t}\psi = \int dt \int d{\bf x}
\psi^{\dag}(t,{\bf x})\partial_{t}\psi(t,{\bf x}) \]
или
\[  \psi_a^{\dag}\partial_{t}\psi_a = \sum_{a=1}^{n}\int dt \int d{\bf x}
\psi_a^{\dag}(t,{\bf x})\partial_{t}\psi_a(t,{\bf x}). \]

Теоретико-полевая формулировка означает, что статистические средние случайных величин
в исходных стохастических задачах могут быть представлены как функциональные интеграллы
с полным набором полей с весом $\exp {\S}(\Phi)$. А это есть ничто иное, как функции Грина
теоретико-полевых моделей с действиями (\ref{actionA}, \ref{actionG}, \ref{actionP}). Для примера,
линейная функция отклика задач (\ref{eq:Langeven}, \ref{eq:Langeven_potts}, \ref{eq:zeta_A}, \ref{eq:zeta_G}, \ref{eq:zeta_P}) 
задается функцией Грина:

\begin{eqnarray}
G=\langle \psi(t, {\bf x}) \psi^{\dag}(t', {\bf x'} ) \rangle =
\int {\D}\psi \int {\D} \psi^{\dag}\ \,
\psi(t, {\bf x}) \psi^{\dag}(t', {\bf x'})\, \exp {\S}(\psi,\psi^{\dag})
\label{respd}
\end{eqnarray}

для каждой конкретной модели. Модель (\ref{actionG}) подразумевает стандартную Фейнмановскую
диаграмную технику с одним пропагатором $\langle\psi \psi^{\dag}\rangle_0$ и двумя тройными вершинами
$ ~ (\psi^{\dag})^2 \psi$, $\psi^{\dag} \psi^2$. В импульсно-временном и частотно-импульсном представлении
пропагатор имеет вид

\begin{eqnarray}
\langle \psi \psi^{\dag}\rangle_0(t,k) = \theta(t)exp\{-\lambda_0(k^2+\tau_0)t\}, \nonumber \\ 
\langle \psi \psi^{\dag}\rangle_0(\omega,k) = \frac{1}{-i\omega + \lambda_0(k^2+\tau_0)}.
\label{psi_psi_dag_AG}
\end{eqnarray}

$\theta(...)$- функция Хевисайда, из-за чего пропагатор является запаздывающим. 
Вследствии этого при анализе диаграмм становится ясно, что все функции Грина, которые зависят только от полей $\psi$ или $\psi^{\dag}$, обязательно имеют замкнутые циклы, и поэтому все такие функции Грина равны нулю.

Для функций $\langle \psi^{\dag}... \psi^{\dag}\rangle$ этот факт является общим следствием причинности, которая справедлива для любой стохастической модели; см. например, обсуждение в \cite{Book3}.

Обнуление функций $\langle \psi...\psi \rangle$ в модели (\ref{actionG}) может рассматриваться как следствие симметрии

\begin{eqnarray}
\psi(t,\x) \rightarrow \psi^{\dag} (-t,-\x), \
\psi^{\dag}(t,\x)\rightarrow\psi(-t,-\x), \
g_0 \rightarrow -g_0.
\label{symm_G}
\end{eqnarray}

Обнуление константы $g_0$ на самом деле несущественно, поскольку, как легко видеть, в модели (\ref{actionG}) фактический параметр разложения в теории возмущении $g_0^2$, а не $g_0$. В дальнейшем мы будем обозначать его как $u_0 \equiv g_0^2$.

В дополнении к (\ref{psi_psi_dag_AG}) диаграмная техника модели A включает в себя пропагатор $\langle \psi \psi \rangle_0$, который имеет вид

\begin{eqnarray}
\langle \psi \psi\rangle_0(t,k) = \frac{1}{k^2+\tau_0}exp\{-\lambda_0(k^2+\tau_0)|t|\}, \nonumber \\ 
\langle \psi \psi\rangle_0(\omega,k) = \frac{2\lambda}{\omega^2 + \lambda_0^2(k^2+\tau_0)^2},
\label{psi_psi_A}
\end{eqnarray}

и одну тройную вершину $\sim \psi \psi^3$.

Модель Поттса аналогичным образом содержит два пропагатора $\langle \psi \psi^{\dag} \rangle_0, \langle \psi \psi \rangle_0$, но в отличии от скалярной модели A, в числителях появляется свертка по индексам полей - символ Кронекера $\delta_{ab}$

\begin{eqnarray}
\langle \psi_a \psi_b^{\dag}\rangle_0(\omega,k) = \frac{\delta_{ab}}{-i\omega + \lambda_0(k^2+\tau_0)}, \nonumber \\ 
\langle \psi_a \psi_b\rangle_0(\omega,k) = \frac{2\lambda\delta_{ab}}{\omega^2 + \lambda_0^2(k^2+\tau_0)^2}.
\label{propogators_P}
\end{eqnarray}

В дополнении к (\ref{propogators_P}) диаграмная техника модели (\ref{actionP}) содержит тройную вершину $\sim \psi^{\dag} \psi^2$.

Галилеево инвариантное взаимодействие с полем скорости $\bfv={v_i(t,x)}$ для сжимаемой жидкости $(\partial_i v_i \neq 0)$ вводится путем замены

\begin{eqnarray}
\partial_t \psi \rightarrow \partial_t\psi +a_0 \partial_i (v_i \psi)+(a_0-1)(v_i \partial_i)\psi=
\nabla_t \psi+a_0(\partial_i v_i)\psi
\label{add_v}
\end{eqnarray}
в (\ref{eq:Langeven},\ref{eq:Langeven_potts}). Где $\nabla_t\equiv\partial_t+v_i\partial_i$ - Лагранжева производная,
$a_0$ - произвольный параметр и $\partial_i=\partial/\partial x_i$. Как мы увидем позже, присутствие нелинейности в 
(\ref{eq:Langeven},\ref{eq:Langeven_potts}) обязывает нас включить оба чена из (\ref{add_v}) для обеспечения мультипливативной перенормируемости.

В реальной проблеме поле $v(t,x)$ описывается с помощью уравнения Навье-Стокса. Однако, мы будем использовать изветсную
модель Обухова-Крейчнана, которая верно передает основные черты реальной развитой турбулентности \cite{FGV}. 
В этой модели скорость подчиняется Гауссовскому распределению с нулевым средним и задается корреляционной функцией

\begin{eqnarray}
\langle v_i(t,x) v_j(t',x')\rangle = \delta(t-t')D_{ij}(\bfr),\ \bfr=x-x'
\label{v_v_correlator}
\end{eqnarray}

где 

\begin{eqnarray}
D_{ij}(\bfr) = D_0\int_{k>m} \frac{d\k}{(2\pi)^d}
\frac{1}{k^{d+\xi}}\lbrace P_{ij}(\k)+\alpha Q_{ij}(\k)\rbrace exp(i\k\bfr).
\label{D_kernel} 
\end{eqnarray}

$P_{ij}(\k)=\delta_{ij}-k_ik_j/k^2$, $Q_{ij}(\k)=k_ik_j/k^2$ - поперечный и продольный проекторы, $k\equiv|\k|$ - волновое число, $D_0>0$ - множитель в амплитуде и $\alpha>0$ - произвольный параметр.
Случай, когда $\alpha=0$, соответствует несжимаемой жидкости ($\partial_i v_i=0$). Показатель $0<\xi<2$ - произвольный параметр; "Колмогоровское" \ значение $\xi=4/3$. Обрезание в интеграле (\ref{D_kernel}) снизу $k=m$, где $m\equiv1/L$- величина, обратная масштабу турбулентности $L$, обеспечивающая ИК-регуляризацию. Его точная форма не имеет значения; просто такое обрезание является простейшим выбором для практических расчетов. Функционалы действия с полным набором полей $\Phi=\lbrace \psi, \psi^{\dag},v \rbrace$:


\begin{eqnarray}
\S(\psi,\psi^{\dagger},v) =  \psi^{\dagger}
\left(-\nabla_{t}+\lambda_{0} \partial^{2}- \lambda_{0}\tau_{0}-a_0(\partial_i v_i)\right)\psi +
\nonumber \\
+ \lambda_{0} (\psi^{\dagger})^2 - u_{0} \psi^{\dagger}\psi^3/3! + \S(\bfv)
\label{actionAT}
\end{eqnarray}
для модели {\it A},
\begin{eqnarray}
\S(\psi,\psi^{\dag},v) =  \psi^{\dag}
\left(-\nabla_{t}+\lambda_{0} \partial^{2}- \lambda_{0}\tau_{0}-a_0(\partial_i v_i)\right) \psi+ 
\nonumber \\
+\frac{g_{0}\lambda_{0}}{2} \left\{ (\psi^{\dagger})^2\psi -
\psi^{\dagger}\psi^2  \right\} + \S(\bfv)
\label{actionGT}
\end{eqnarray}
для процесса Грибова,
\begin{eqnarray}
\S(\psi,\psi^{\dagger},v) =  \psi^{\dagger}_a
\left(-\nabla_{t}+\lambda_{0} \partial^{2}- \lambda_{0}\tau_{0} -a_0(\partial_i v_i)\right)
\psi_a +
\nonumber \\
+\lambda_{0} \psi_a^{\dagger} \psi_a^{\dagger}-\,
\frac{g_{0}\lambda_0}{2} R_{abc}\psi_a^{\dagger}\psi_b\psi_c + \S(\bfv)
\label{actionPT}
\end{eqnarray}
для модели Поттса получаются из (\ref{actionA}, \ref{actionG},\ref{actionP})  путем замены (\ref{add_v}) и добавлением члена, соответствующего Гауссовскому полю $\bfv$ с коррелятором:
\begin{eqnarray}
\S(\bfv) = -\frac{1}{2}\int dt \int dx \int dx' v_i(t,x) D^{-1}_{ij}(\bfr)v_j(t,x'),
\label{actionV}
\end{eqnarray}
где
\[
D^{-1}_{ij}(\bfr) \propto D^{-1}_0 \bfr^{-2d-\xi}
\]
является ядром обратной линейной операции для функции $D_{ij}(\bfr)$ из (\ref{D_kernel}).

Теперь, при расширении наших функционалов действия, в Фейнмановской диаграмной технике появляются новые пропагатор $\langle vv\rangle_0$, задающийся соотношением (\ref{v_v_correlator}) и новая вершина:
\begin{eqnarray}
\psi^{\dag}v_iV_i\psi \equiv -\psi^{\dag}\lbrace(v_i\partial_i)\psi+a_0(\partial_i v_i)\rbrace\psi.
\label{vpsipsi}
\end{eqnarray}
Для векторных полей Поттса вид вершины легко обобщается:
\begin{eqnarray}
\psi^{\dag}_av_iV_{i,ab}\psi_b \equiv -\psi^{\dag}_a\lbrace(v_i\partial_i)\psi_a+a_0(\partial_i v_i)\rbrace\psi_a.
\label{vpsipsi_index}
\end{eqnarray}
Для диаграмм это подразумевает добавление в вершину множителей
\begin{eqnarray}
V_i=-ik_i-ia_0q_i, \ V_{i,ab}=-i\delta_{ab}(k_i+a_0q_i)
\label{vpsipsi_vertex}
\end{eqnarray}
соответственно, где $k_i$ - импульс поля $\psi$, а $q_i$- импульс поля $v_i$.

Симметрия $\psi, \psi^{\dag} \rightarrow -\psi, -\psi^{\dag}$ в модели {\it A} сохраняется и в
полной модели (\ref{actionAT}), в то время как для полной модели (\ref{actionAT}) к соотношениям (\ref{symm_G}) следует добавить преобразование
\begin{eqnarray}
a_0 \rightarrow (1-a_0),
\label{symm_G_add}
\end{eqnarray}
что легко проверить в (\ref{actionGT}) путем интегрирования по частям.

В полных моделях роль констант связи (параметров разложения в теории возмущения) играют три параметра
\begin{eqnarray}
u_0=g_0^2,\ \omega=D_0/\lambda_0 \propto \Lambda^{\xi},\ \omega_0 a_0 \propto \Lambda^{\xi}.
\label{charges}
\end{eqnarray}
Последние соотношения вытекают из соображений размерности (см. следующий раздел). Оказывается, как мы увидим дальше, модели Грибова и {\it A} логарифмичны при $d=d_c\equiv 4,\ u_0\propto \Lambda^{4-d}$, а модель Поттса при $d=d_c\equiv 6,\ u_0\propto \Lambda^{6-d}$ (в этом случае взаимодействия $\psi^{\dag}\psi^3,\ (\psi^{\dag})^2\psi,\ \psi^{\dag}\psi^2$ будут безразмерными). 
Таким образом, для моделей с одним зарядом (\ref{actionA}, \ref{actionG}, \ref{actionP}), значение $d=d_c$ является верхней критической размерностью, и отклонение $\varepsilon=d_c-d$ играет роль формального параметра разложения в РГ подходе: критические индексы вычисляются как ряды по $\varepsilon$.

Взаимодействие $\propto \psi^{\dag}v\partial \psi$ в полных моделях (\ref{actionAT}, \ref{actionGT}, \ref{actionPT}) становится логарифмичным, когда $\xi=0$. Параметр $\xi$ не связан с пространственной размерностью. Однако, для РГ анализа полных моделей важно, чтоб все взаимодействия были логарифмичными. В противном случае, одно из них будет слабее чем другое с РГ точки зрения, и тем самым не будет влиять на ИК поведение в главном порядке.
В результате, некотрые из скейлинговых режимов полной модели будут упущены.
Для того, чтобы изучить все возможные классы универсальности, нам необходимо, чтобы в нашей теории все взаимодействия рассматривались на равных основаниях.
Таким образом, мы будем обращаться с $\varepsilon$ и $\xi$ как с малыми параметрами одинакового порядка $\varepsilon \propto \xi$. 
Вместо простого $\varepsilon$- разложения в однозарядной теории, координаты фиксированных точек, критические размерности и другие величины будут вычисляться в виде двойного разложения в $\varepsilon-\xi$ - плоскости вблизи начала координат, то есть вокруг точки, в которой все константы (\ref{charges}) безразмерны. Аналогичная ситуация была и раньше в различных моделях турбулентности и критического поведения, например \cite{AHH, Alexa, AIK, AIM, Sak}.











%%%%%%%%%%%%%%%%%%%%%%%%%%%%%%%%%%%%%%%%%%%%%%%%%%%%%%%%%%%%%%
% Литература                                                 %
%%%%%%%%%%%%%%%%%%%%%%%%%%%%%%%%%%%%%%%%%%%%%%%%%%%%%%%%%%%%%%
\section*{References}
\begin{thebibliography}{99}

\bibitem{Zinn} Zinn-Justin J 1989 {\it Quantum Field Theory and Critical
Phenomena} (Oxford: Clarendon)

\bibitem{Book3} Vasil'ev A N 2004 {\it The Field Theoretic Renormalization
Group in Critical Behavior Theory and Stochastic Dynamics}
(Boca Raton: Chapman \& Hall/CRC)

\bibitem{Ashkin} Ashkin J and Teller E 1943 {\it Phys. Rev.}
{\bf 64} 178;\\
Potts R B 1952
{\it Proc. Camb. Phil. Soc.}
{\bf 48} 106

\bibitem{Baxter} Baxter R J 1973 {\it J.Phys. C: Solid St. Phys}
{\bf 6} L445

\bibitem{Golner} Golner G R 1973 {\it Phys. Rev.} A
{\bf 8} 3419

\bibitem{Zia} Zia R K P and Wallace D J 1975 {\it J. Phys. A: Math. Gen.}
{\bf 8} 1495

\bibitem{Priest} Priest R G and Lubensky 1976 {\it Phys. Rev.} B
{\bf 13} 4159; Erratum: B {\bf 14} 5125(E)

\bibitem{Amit} Amit D J 1976 {\it J. Phys. A: Math. Gen}
{\bf 9} 1441


\bibitem{Alcantara} 
de Alcantara Bonfim O F, Kirkham J E and McKane A J 1980
{\it J. Phys. A: Math. Gen}
{\bf 13} L247;\\
de Alcantara Bonfim O F, Kirkham J E and McKane A J 1981
{\it J. Phys. A: Math. Gen}
{\bf 14} 2391


\bibitem{Cardy} 
Cardy J 2009
{\it J. Stat. Phys.}
{\bf 137} 814;\\
Ikhelf Y and Cardy J 2009
{\it J. Phys. A: Math. Theor.}
{\bf 42} 102001

\bibitem{IMConform} 
International Meeting {\it Conformal Invariance, Discrete Holomorphicity and Integrabil-
ity } (Helsinki, 10–16 June 2012). https://wiki.helsinki.fi/display/mathphys/cidhi2012


\bibitem{Ivanov2008} 
Ivanov D Yu 2008 {\it Critical Behaviour of Non-Ideal Systems} (Weinheim, Germany: Wiley-VCH)

\bibitem{Lacasta} 
Lacasta A M, Sancho J M and Sagu\'{e}s F 1995
{\it Phys. Rev. Lett.} 
{\bf 75} 1791;\\
Berthier L 2001 
{\it Phys. Rev.} E 
{\bf 63} 051503;\\
Berthier L, Barrat J-L and Kurchan J 2001 
{\it Phys. Rev. Lett.}
{\bf 86} 2014;\\
Berti S, Boffetta G, Cencini M and Vulpiani A 2005 
{\it Phys. Rev. Lett.}
{\bf 95} 224501

\bibitem{Chan} 
Chan C K, Perrot F and Beysens D 1988
{\it Phys. Rev. Lett.}
{\bf 61} 412;\\
Chan C K, Perrot F and Beysens D 1989
{\it Europhys. Lett.}
{\bf 9} 65;\\
Chan C K 1990 
{\it Chinese J. Phys.}
{\bf 28} 75;\\
Chan C K, Perrot F and Beysens D 1991
{\it Phys. Rev. A.}
{\bf 43} 1826

\bibitem{HH} Halperin B I and Hohenberg P C 1977 {\it Rev. Mod. Phys.}
{\bf 49} 435;\\ Folk R and Moser G 2006 {\it J. Phys. A: Math. Gen.}
{\bf 39} R207

\bibitem{Hinr} Hinrichsen H 2000 {\it Adv. Phys.} {\bf 49} 815;\\
 \'Odor G 2004 {\it Rev. Mod. Phys.} {\bf 76} 663

\bibitem{JT} Janssen H-K and T\"{a}uber U C 2004 {\it Ann. Phys. (NY)}
{\bf 315} 147

\bibitem{Ivanov} Ivanov D Yu 2003 {\it Critical Behaviour of
Non-Idealized Systems} (Moscow: Fizmatlit) [in Russian]

\bibitem{quench} Khmel'nitski D E 1975 {\it Sov. Phys. JETP} {\bf 41} 981;\\
Shalaev B N 1977 {\it Sov. Phys. JETP} {\bf 26} 1204;\\
Janssen H-K, Oerding K and Sengespeick E 1995
{\it J. Phys. A: Math. Gen.} {\bf 28} 6073

\bibitem{Satten} Satten G and Ronis D 1985 {\it Phys. Rev. Lett.}
{\bf 55} 91; 1986 {\it Phys. Rev.} A {\bf 33} 3415

\bibitem{Onuki} Onuki A and Kawasaki K 1980 {\it Progr. Theor. Phys.}
{\bf 63} 122;\\
Onuki A, Yamazaki K and Kawasaki K 1981 {\it Ann. Phys.} {\bf 131} 217;\\
Imaeda T, Onuki A and Kawasaki K 1984 {\it Progr. Theor. Phys.} {\bf 71} 16

\bibitem{Beysens} Beysens D, Gbadamassi M and Boyer L 1979
{\it Phys. Rev. Lett} {\bf 43} 1253;\\
Beysens D and Gbadamassi M 1979 {\it J. Phys. Lett.} {\bf 40} L565

\bibitem{Ruiz} Ruiz R and Nelson D R 1981 {\it Phys. Rev.} A {\bf 23}
3224; {\bf 24} 2727;\\
Aronowitz A and Nelson D R 1984 {\it Phys. Rev.} A {\bf 29} 2012



\bibitem{AHH} Antonov N V, Hnatich M and Honkonen J 2006
{\it J. Phys. A: Math. Gen.} {\bf 39} 7867

\bibitem{Alexa} Antonov N V and Ignatieva A A 2006
{\it J. Phys. A: Math. Gen.} {\bf 39} 13593

\bibitem{AIK} Antonov N V, Iglovikov V I and Kapustin A S 2009
{\it J. Phys. A: Math. Theor.} {\bf 42} 135001

\bibitem{AIM} Antonov N V, Ignatieva A A and Malyshev A V 2010
E-print LANL arXiv:1003.2855 [cond-mat]; to appear in {\it PEPAN
(Phys. Elementary Particles and Atomic Nuclei, published by JINR)} {\bf 41}

\bibitem{AKM}
Antonov N V, Kapustin A S and Malyshev A V 2011 
{\it Theor. Math. Phys.} 
{\bf 169} 1470


\bibitem{AM}
Antonov N V and Malyshev A V 2011 
{\it Theor. Math. Phys.}
{\bf 167} 444

\bibitem{AM1}
Antonov N V and Malyshev A V 2012 
{\it J. Phys. A: Math. Theor.} 
{\bf 45} 255004


\bibitem{AK}
Antonov N V and Kapustin A S 2010 
{\it J. Phys. A: Math. Theor.}
{\bf 43} 405001


\bibitem{FGV} Falkovich G, Gaw\c{e}dzki K and Vergassola M  2001
{\it Rev. Mod. Phys.} {\bf 73} 913

\bibitem{JphysA} Antonov N V 2006 {\it J. Phys. A: Math. Gen.} {\bf 39} 7825

\bibitem{Compress} Antonov N V, Hnatich M and Nalimov M Yu 1999
{\it Phys. Rev.} E {\bf 60} 4043

\bibitem{vanK} van Kampen N G 2007 {\it Stochastic Processes in Physics
and Chemistry, 3rd ed.} (Amsterdam: North Holland)

\bibitem{Sak} Sak J 1973 {\it Phys. Rev.} B {\bf 8} 281;\\
Honkonen J and Nalimov M Yu 1989 {\it J. Phys. A: Math. Gen.} {\bf 22} 751;\\
Janssen H-K 1998 {\it Phys. Rev.} E {\bf 58} R2673;\\
Antonov N V 1999 {\it Phys. Rev.} E {\bf 60} 6691;\\
2000 {\it Physica} D {\bf 144} 370

\bibitem{Levy} Janssen H-K, Oerding K, van Wijland F and Hilhorst H J
1999 {\it Eur. Phys. J.} B {\bf 7} 137; Janssen H-K and Stenull O 2008
{\it Phys. Rev.} E {\bf 78} 061117

\end{thebibliography}
\underline{Структура и объем работы}. Диссертация состоит из введения, 6 глав, заключения и списка литературы из 129 наименований. Работа изложена на 174 страницах и содержит 51 рисунок и 1 таблицу.

\end{document}
